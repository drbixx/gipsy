\documentstyle[rep10]{report}
\spaceskip=3.33pt plus 3.33pt minus 1pt
\font\tt=cmtt10
\begin{document}
\begin{titlepage}
\null
\begin{center}
{\Large Universiteit Groningen -- Kapteyn Laboratorium}\\
\vfill
{\huge SHELTRAN\\}

\end{center}
\end{titlepage}
\thispagestyle{empty}
\tableofcontents
\newpage
\pagenumbering{arabic}
\chapter{Introduction.}

The program SHLTRN accepts a program written in the SHELTRAN language
and generates the corresponding FORTRAN 77 program and, optionally, a
source listing. 

SHELTRAN is 'structured FORTRAN' - this being ANSI FORTRAN 77 with
certain facilities replaced by corresponding 'structured' facilities,
plus additional facilities with no FORTRAN equivalents. 

SHELTRAN declares statement labels (except for the FORMAT statement)
illegal, and hence any control statement requiring statement labels as
arguments is also illegal.  These statements are replaced by
closely-matching 'structured' facilities with controlled exits. 
SHELTRAN also provides various forms of the repetition construct
(allowing controlled exits). 

\chapter{Description of the language.}

In describing the SHELTRAN language the following nomenclature is used:

\begin{itemize}

\item SHELTRAN keywords appear in capitals.  (SHELTRAN keywords always
start on a new line and must be followed by at least one blank)

\item Syntactic elements are enclosed within $<$ $>$. 

\item Items enclosed within [ ] are optional. 

\item $<$statement$>$ is used to denote any SHELTRAN and/or FORTRAN
statement(s). 

\end{itemize}

\section{Statement labels.}

In SHELTRAN the statement label is forbidden for all executable
statements.  The only context in which the statement label is permitted
is in combination with a FORMAT statement.  In this case the label must
be less than 10000. 

As a direct consequence the following normal FORTRAN statements become
unuseable in SHELTRAN:

\begin{itemize}

\item ASSIGN

\item all forms of GOTO

\item all forms of IF

\item DO

\item the RETURN $<$variable$>$ form of the RETURN statement

\item the use of statement numbers in END= or ERR= clauses in I/O
statements

\item the use of statement numbers as arguments of CALL statements

\end{itemize}

To replace FORTRAN control structures several new SHELTRAN structures
have been introduced. 

\section{IF construct.}
\subsection{Form:}
\begin{verbatim}
      IF <logical expression>
      THEN
         <statement>... 
    [ ELSEIF <logical expression>
      THEN
         <statement>...      ] 
    [ ELSE
         <statement>...      ] 
      CIF 
\end{verbatim}
\subsection{Notes:}
\begin{itemize}

\item $<$logical expression$>$ represents any valid FORTRAN logical
expression which can take a value .TRUE.  or .FALSE.  It need not be
surrounded by parentheses. 

\item ELSEIFs when present must precede any corresponding ELSE. 

\item The ELSE clause is optional. 

\item Instead of CIF ENDIF may also be used. 

\end{itemize}
\subsection{Actions:}

When the IF or ELSEIF condition has the value .TRUE., the statements
following are executed and control is transferred to the statement
following the CIF. 

When the value is .FALSE., control is passed to the next ELSEIF and this
$<$logical expression$>$ is evaluated. 

When the IF and all ELSEIF clauses have a value which is .FALSE.,
control is passed to the statement following the corresponding ELSE or
CIF. 

\subsection{Examples:}
\begin{verbatim}
      IF KEY.EQ.K1 
      THEN 
         KODE=1 
      ELSEIF KEY.EQ.K2 
      THEN 
         KODE=2 
      ELSE 
         KODE=-1
      CIF


      IF X.LT.0 
      THEN
         X=999.0 
      CIF 
\end{verbatim}
\section{SELECT construct.}
\subsection{Form:}
\begin{verbatim}
      SELECT <arithmetic expression>
      CASE n1 [,n2...]
    [    <statement>...         ] 
    [ CASE m1 [,m2...]
         <statement>...         ] 
      OTHER 
    [    <statement>...         ] 
      CSELECT 
\end{verbatim}
\subsection{Notes:}
\begin{itemize}

\item n1 [,n2...] represents a series of unsigned integer values,
separated by commas.  At least one integer must be present for each CASE
statement. 

\item the integer values may appear in any order in the CASE clause. 

\item the CASE clauses need not contain all values in the range 1 to N,
where N is the largest integer value used.  However, a specific value
may only appear once. 

\item the largest value used in a SELECT construct must not exceed 198. 

\item the OTHER clause is mandatory. 

\end{itemize}
\subsection{Actions:}

The expression after the SELECT is evaluated and truncated to an integer
value.  Control is transferred to the statements following the CASE
statement which contains the value of the SELECT expression.  If no such
value can be found, the OTHER clause is selected.  Finally control is
transferred to the statement after CSELECT. 

\subsection{Example:}
\begin{verbatim}
      SELECT COLOR+1.0
      CASE 1
         COLTXT='BLACK'
      CASE 2
         COLTXT='RED'
      CASE 3
         COLTXT='GREEN'
      CASE 5
         COLTXT='BLUE' 
      CASE 8
         COLTXT='WHITE'
      OTHER 
         COLTXT='MIXTURE'
      CSELECT 
\end{verbatim}

\section{WHILE construct.}
\subsection{Form:}
\begin{verbatim}
      WHILE <logical expression>
         <statement>...
    [    XWHILE 
         <statement>...      ] 
      CWHILE 
\end{verbatim}
\subsection{Actions:}

Whilst $<$logical expression$>$ has a value .TRUE., the statements
within the loop WHILE-CWHILE will be executed, and $<$logical
expression$>$ re-evaluated.  When $<$logical expression$>$ has a value
.FALSE., control is transferred to the statement following CWHILE.  At
any point in the WHILE loop an XWHILE statement may be inserted.  This
statement transfers control to the statement following CWHILE
independently of the value of $<$logical expression$>$. 

\subsection{Example:}
\begin{verbatim}
      
      WHILE I.LE.72 
         CHR=LINE(I) 
         IF CHR.EQ.BLANK 
         THEN
            I=I+1 
         ELSE
            IF CHR.GE.'A'.AND.CHR.LE.'Z'
            THEN
               XWHILE
            CIF 
         CIF 
      CWHILE

\end{verbatim}
\section{REPEAT construct. }
\subsection{Form:}
\begin{verbatim}
      
      REPEAT 
         <statement>...
    [    XREPEAT
         <statement>...  ] 
      UNTIL <logical expression>

\end{verbatim}
\subsection{Actions:}

The statements between REPEAT and UNTIL are executed, and $<$logical
expression$>$ evaluated.  When $<$logical expression$>$ has the value
.TRUE., control is transferred to the the statement after UNTIL; if it
is .FALSE., the statement after REPEAT gains control.  At any point in
the REPEAT loop an XREPEAT statement may be inserted.  This statement
transfers control to the statement following UNTIL, independently of
$<$logical expression$>$. 

\subsection{Example:}
\begin{verbatim}

     REPEAT
        CALL WAIT(EVENT)
        SELECT EVENT
        CASE 1
           CALL TYPEIN 
        CASE 2,3,7,4
           CALL SERVE
        OTHER 
           FINISH=.TRUE. 
        CSELECT 
     UNTIL FINISH

\end{verbatim}
\section{FOR construct.}
\subsection{Form:}
\begin{verbatim}

      FOR <loop control>
         <statement>...
    [    XFOR 
         <statement>...  ] 
      CFOR 

\end{verbatim}
\subsection{Note:}

The rules for writing $<$loop control$>$ are identical to those for the
FORTRAN DO-loop. 

\subsection{Actions:}

The actions follow the FORTRAN DO-loop control actions.  At any point in
the FOR-loop an XFOR statement may be inserted.  This statement
transfers control to the statement following CFOR. 

\subsection{Example:}
\begin{verbatim}
 
      FOR I=1,80
         LINE(I)=' ' 
      CFOR

\end{verbatim}
\section{PERFORM and PROC.}
\subsection{Form:}
\begin{verbatim}

C     To invoke a procedure section: 
      PERFORM <procedure name> 
C     To define a procedure section: 
      PROC <procedure name>
         <statement>... 
      CPROC 

\end{verbatim}
\subsection{Notes:}
\begin{itemize}

\item The procedure name must not be longer than 31 characters. 

\item The statements in a PROC may not include:

\begin{itemize}

\item STOP or RETURN

\item XWHILE, XREPEAT or XFOR unless the corresponding loop is wholly
included within the range of the PROC-CPROC construct. 

\end{itemize}

\item A procedure section may only be defined in a program after all
PERFORMs to that PROC have occurred, i.e.  PROCs normally preceed the
END statement. 

\item A PERFORM in a FOR-loop causes non-standard FORTRAN to be
generated. 

\end{itemize}
\subsection{Actions:}

Whenever a PERFORM statement is encountered, the appropriate statements
between PROC and CPROC are executed. 

\subsection{Example:}
\begin{verbatim}

         . 
         . 
      PERFORM CLFILE
         . 
         . 
         . 
         .
      PROC CLFILE 
         FOR LUN=1,NUNITS
            CLOSE(UNIT=LUN) 
         CFOR
      CPROC 

\end{verbatim}
\section{I/O statements.}
\subsection{Form:}
\begin{verbatim}

      ...,ERR=<exit specifier>,... 
      and/or 
      ...,END=<exit specifier>,... 

\end{verbatim}
\subsection{Notes:}
\begin{itemize}

\item The following I/O statements are recognized: READ, WRITE, OPEN,
CLOSE, REWIND, BACKSPACE, INQUIRE, ENDFILE. 

\item $<$exit specifier$>$ can be any one of the following: XWHILE,
XREPEAT, XFOR, STOP, RETURN or $<$procedure name$>$. 

\item The exit specification must appear on the first line of the I/O
statement. 

\end{itemize}
\subsection{Actions:}

The use of $<$procedure name$>$ is equivalent to a PERFORM of the
procedure between the I/O statement and the statement following it.  The
use of the other items will have the normal SHELTRAN meaning. 

\subsection{Example:}
\begin{verbatim}

      WHILE .TRUE.
         READ(1,FMT='(A)',END=XWHILE,ERR=ERRPRC) LINE
         WRITE(2,FMT='(1X,A)',ERR=ERRPRC) LINE 
      CWHILE
         .
         .
         .
      PROC ERRPRC 
         REWIND 1
         REWIND 2
      CPROC 

\end{verbatim}
\section{CALL statement.}

Within the CALL argument list statement label arguments may be replaced
by one of the same keywords for I/O statements.  Such arguments must be
preceded by an asterisk (*).  E.g. 

\begin{verbatim}

      CALL sub ( ..., *XWHILE, ... )

\end{verbatim}

\begin{center}
Notes:
\end{center}

\begin{itemize}

\item This feature can only be used in calls to existing FORTRAN
subroutines that specify alternate returns.  The designers of SHELTRAN
consider this feature undesirable and only provided an interface to
existing routines, but have intentionally omitted the facility to
construct them, i.e.  the 'RETURN i' statement is not allowed. 

\item Asterisk arguments must appear on the first line of the CALL
statement. 

\end{itemize}
\section{Additional features.}
\subsection{Comment lines.}

Comment lines follow the same rules as FORTRAN. 

\subsection{Note lines.}

Note lines are identified by an N in column 1, the text area being
columns 2 to 72.  In the line printer listing the text of a Note line is
printed in columns 75 to 124 and on the same line as the statement
following the Note line.  If the next line is also a Note, the text is
printed in the same columns of an otherwise blank line.  The text of the
Note line may be 50 characters long and may appear anywhere in columns 2
to 72.  Leading blanks between column 1 and the start of the Note text
are ignored, i.e.  in the listing Notes are aligned. 

\subsection{Eject lines.}

Eject lines are identified by an E in column 1, the text area being
columns 7 to 72.  The action of an Eject line is to force a new page on
the line printer listing, with the page header taken from the Eject line
text. 

\subsection{Fortran directive lines.}

Fortran directive lines are identified by an F in column 1.  The
remainder of the line is written in columns 1 to 79 of the FORTRAN
output file.  This facility can be used to pass directives to the
FORTRAN compiler (e.g.  debug directives).  It should not be used to
bypass SHELTRAN control structures. 

\subsection{Input source deck.}

The input SHELTRAN source deck follows the normal FORTRAN layout rules,
apart from the above-mentioned features.  No indentation of structures
need to be included, as the pre-processor will automatically provide
indentation in the listing. 

\subsection{Output source deck.}

The output source deck consists of the input source deck plus the
necessary generated FORTRAN statements.  It contains no Comment, Eject
or Note lines.  The output deck is not indented. 

\subsection{Include Files.}

Code from other SHELTRAN sources can be included by putting an I in
column 1 followed by the name of the source to be included.  The code
included in this way will also be processed by the SHELTRAN compiler. 

\section{Limitations.}
\begin{itemize}

\item The maximum number of procedure sections in a program unit is 99. 

\item The following variable names are reserved by SHELTRAN and must not
be used: IF, ELSEIF, WHILE, UNTIL, FOR; all 6-character variable names
beginning with 'IZZZ'

\item It is not allowed to give variable names beginning with 'I' an
implicit type other than INTEGER. 

\item PERFORM statements in FOR-loops cause non-standard FORTRAN to be
generated.  This is because after a procedure section has been executed,
a jump into the generated DO-loop is necessary to resume execution after
the PERFORM statement.  Most compilers however allow this kind of jump
which was part of the previous FORTRAN standard (FORTRAN IV). 

\end{itemize}

\chapter{Update history.}
\begin{verbatim}

 Jan-83  version 5.0: completely revised for FORTRAN 77
                         J.P. Terlouw 

21-Mar-83: scratch file opening changed; instead of 
      TAPE2 and TAPE3 we now use SCRAT01 and SCRAT02
                       J.P.T.

 3-Apr-84: error message for non-standard Fortran when using
      PERFORM in for loops suppressed. 
                       J.P.T.
 
 1-Nov-88: bug in FOR loop control repaired, procedure names
      upto 31 characters, no trailing blanks in target and
      source listing.
                       K.G.B.

\end{verbatim}
\end{document}
