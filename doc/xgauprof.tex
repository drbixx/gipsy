\documentclass[10pt,a4paper]{report}
%\documentclass[11pt,fleqn]{article}
\usepackage{graphicx}
\def\eherm{A{e^{-\frac{1}{2} {\left(\frac{\nu - b}{c}\right)}^2}}}
\def\epow{A{e^{-\frac{1}{2} {\left(\frac{x - x_0}{\sigma}\right)}^2}}} 
\def\erf{\rm erf}
\def\erfc{\rm erfc}
\def\ampfac{\left(1+c_0\,h_4+c_2\,h_4+3c_4\,h_4\right)}

%\parskip=0.8cm

%\vspace{12pt}

\oddsidemargin=0cm
\evensidemargin=0cm
\textwidth=16cm

\begin{document}

\begin{center}
{\Large XGAUPROF}\\[1.5cm]
{\large A GIPSY program to fit parameters of a Gauss function, 
the Gauss-Hermite series or 
a Voigt line-shape to data from GIPSY sets or ASCII files }\\[1.5cm]
M.G.R. Vogelaar\\
May 7,  1999, revision Mar 25, 2012
\end{center}

\begin{figure}[htb]
  \centering
  \includegraphics[width=12cm, angle=0]{gui.pdf}
  \caption{\it XGAUPROF Graphical User Interface}
  \label{fig:gui}
\end{figure}  


\begin{flushleft}


\chapter{Introduction} 

\section{History}
An article from 1966 called {\it Computer Analysis of Observed Distributions
into Gaussian components} by Kaper et al. described a first Kapteyn
Institute attempt to use a computer to decompose an observed frequency 
or intensity profile  
into a superposition of Gaussian components. A least squares method was 
applied to get a `best fit' to study profiles of the spectral line at 21 cm. 
The program was implemented on a Stantec ZEBRA, installed in 1956. It had a
memory of 8192 words of 33 bits. This limited the analysis to 150 data points
and 6 parameters. The average access time to an address on the ZEBRA was 
0.005 sec. and one least-squares iteration cycle took about 20 minutes.
Initial guesses for the parameters were found by sketching a decomposition 
into Gaussians on a graph of the observed profile. 
\section{Standard Gauss}
GIPSY program {\bf \it XGAUPROF} finds best fit parameters for three functions
given data from file or from a GIPSY set. The base function is a Gauss. 
The parameters are fitted with a least-squares fit algorithm which needs 
initial guesses. A special routine automates the procedure of getting
reasonable estimates for a Gauss and the other functions in the program.
The user can modify these values by hand
using a graphical slider. 
A one component standard Gauss function with background is defined as:

\begin{equation}
\phi(x) = \epow + Z
\end{equation}

Usually x is frequency or velocity. $A$ is the amplitude at the center 
$x = x_0$ 
and ${\sigma }$ is the dispersion of the profile, i.e. the half width at 
$\frac{A}{\sqrt{e}}$ and Z is a representation for the background.


In {\bf \it XGAUPROF} the background consists of a constant, a linear and 
a quadratic term and
is represented by:
\begin{equation}
Z = Z_0 + Z_1\, (x-x_0) + Z_2\, (x-x_0)^2
\end{equation}
The constant term $Z_0$ is often referred to as 'zero level'. 


\section{Gauss-Hermite series}

If your profile deviates from a Gaussian shape (e.g. asymmetric profiles) 
then you can use the so called {\it Gauss-Hermite} series.
The lowest order term of the series is a
Gaussian. The higher order terms are orthogonal to this Gaussian.
The higher order that we use in our fits are the 
parameters $h_3$ and $h_4$ measuring asymmetric and symmetric 
deviations of a Gaussian. 
The Gauss-Hermite function used in {\bf \it XGAUPROF} is (Van der Marel \& Franx):



\begin{center} 
\begin{figure}[htb]
\begin{minipage}{7cm} 
  \centering
  \includegraphics[height=8cm,width=6cm, angle=0]{h3.pdf}
  \caption{\it Gauss-Hermite $h3 = 0.2$, $h_4 = 0$. } 
  \label{fig:skewness}
\end{minipage}
\begin{minipage}{7cm}
  \centering
  \includegraphics[height=8cm,width=6cm, angle=0]{h4.pdf}
  \caption{\it Gauss-Hermite $h3 = 0$, $h_4 =0.1$. }
  \label{fig:kurtosis}
\end{minipage} 
\end{figure} 
\end{center}


\begin{equation}
\phi(x) = a\,e^{-\frac{1}{2}y^2} \left\{ 1+\frac{h_3}{\sqrt{6}}
(2\sqrt{2}y^3-3\sqrt{2}y) + 
\frac{h_4}{\sqrt{24}}(4y^4-12y^2+3)   \right\} + Z
\end{equation} 


with $y \equiv \frac{x-b}{c}$. 
Note that the parameters $a$, $b$ and $c$ are equivalent to $A$, $x_0$ and
$\sigma$ of a Gaussian, only if $h_3 = h_4 = 0$.


\section{Voigt profile}

The line-shapes of spectroscopic 
transitions depend on the broadening mechanisms
of the initial and final states, and include natural broadening, 
collisional broadening, power broadening, and
Doppler broadening. Natural, collisional, and power broadening are 
homogeneous mechanisms and produce Lorentzian line-shapes. 
Doppler broadening is a form of inhomogeneous broadening and has a 
Gaussian line-shape. Combinations of
Lorentzian and Gaussian line-shapes can be approximated by a Voigt profile. 
In fact, the Voigt profile is a convolution of these two line-shapes. 
It has the form:

\begin{equation}
\phi(\nu) = \frac{A}{\alpha_D }\sqrt{\frac{\ln(2)}{\pi }} \, K(x,y) + Z
\end{equation}


$K(x,y)$ is the {\it Voigt function} and is defined as:
$$
K(x,y) = \frac{y}{\pi} \int_{-\infty}^{\infty} \frac{e^{-t^{2}}}{y^{2}+(x-t)^{2}} dt 
$$

where 
\begin{equation} 
y \equiv \frac{\alpha_L}{\alpha_D}\sqrt{\ln(2)} \hspace{0.5cm} and \hspace{0.5cm}
x \equiv \frac{(\nu-\nu_0)}{\alpha_D} \sqrt{\ln(2)}
\end{equation}

$y$ is the ratio of Lorentz to Doppler widths and $x$ the frequency 
scale (in units of the Doppler Line-shape half-width  $\alpha_D$.


\chapter{The standard Gauss function}
\section{Properties of a profile using moments of a distribution}

One way to characterize a distribution is the use of its moments.
The $k$th-$moment$ of a distribution $\phi(x)$ is:
$$ \mu_k = \int\limits_{-\infty}^{\infty}{(x-x_0)}^k\,\phi(x) dx $$

With these moments one can calculate more familiar properties of a 
distribution like the line strength (i.e. area under the curve) 
$\gamma$, the mean $X_0$,
the dispersion $\sigma$, the coefficient of skewness
$\xi_1 \equiv \mu_3/\mu_2^{3/2}$ and the coefficient of kurtosis 
$\xi_2 \equiv \mu_4/\mu_2^2$. These quantities are calculated from:
$$\gamma \equiv \int\limits_{-\infty}^{\infty} \phi(x)\, dx
\hspace{0.5cm} 
X_0 \equiv \frac{1}{\gamma}\int\limits_{-\infty}^{\infty}x\, \phi(x)\,dx
\hspace{0.5cm} 
\sigma^2_0 \equiv \frac{1}{\gamma}\int\limits_{-\infty}^{\infty} 
{(x-X_0)}^2\,\phi(x)\,dx 
$$
\begin{equation}
\xi_i \equiv \frac{1}{\gamma}\int\limits_{-\infty}^{\infty}
\left((x-X_0)/\sigma\right)^{2+i}\,\phi(x)\,dx \hspace{1cm} (i=1,2) 
\end{equation}

{\it Skewness} is a lack of symmetry in a distribution.
If a distribution is symmetric about its mean it has zero skewness.
The skewness defined above is called the Fisher Skewness.

{\it Kurtosis} is a measure of how "fat" the tails of a distribution are,
measured relative to a normal distribution having the same
standard deviation. A distribution is said to be {\it leptokurtic}
if its tails are fatter than those of a corresponding normal distribution
(high peak).
It is said to be {\it platykurtic} if its tails are thinner than those of the
normal distribution (flat-topped curve). The Fisher Kurtosis is defined by
\begin{equation}
\framebox{$\xi_f = \xi_2 - 3$}  
\end{equation} 
From here we will use $\xi_f$ if we talk about kurtosis.

Using the expressions above, one finds for our standard Gaussian:

\medskip
{\bf The integrated line strength $\gamma $:}
$$ \gamma = \int\limits_{-\infty}^{\infty} \epow dx =
 A\,\sigma\, \sqrt{\frac{\pi}{2}}\, 
\erf {\left(\frac{x-x_0}{\sigma\sqrt{2}}\right)} \Bigr|_{-\infty}^{\infty}$$

$erf$ is the 'error'-function and $\erf (-\infty) = -1$ and $\erf (\infty)= 1.$\\
Then:
\begin{equation}
\framebox{$\gamma = A\sigma\sqrt{2\pi}$}
\end{equation} 

\vspace{1cm} 
{\bf The mean abscissa $X_0$:} 
\begin{eqnarray*}
X_0 &=& \frac{1}{\gamma}\int\limits_{-\infty}^{\infty} x\epow dx\\
&=& \frac{1}{\gamma}  \left[ - \epow + 
    A\sqrt{\frac{\pi}{2}}\,x_0\, \sigma \, 
    \erf \left(\frac{x-x_0}{\sigma\sqrt{2}}\right) \right]
                    \biggr|_{-\infty}^{\infty}\\
&=& \frac{2}{\gamma}A\, \sigma \,\sqrt{\frac{\pi}{2}}\,x_0 = x_0
\end{eqnarray*} 

\begin{equation}
\framebox{$X_0 = x_0$}
\end{equation}

\vspace{1cm} 
{\bf The dispersion $\sigma_0$:}
\begin{eqnarray*}
\sigma^2_0 &=& \frac{1}{\gamma}\int\limits_{-\infty}^{\infty} (x-x_0)^{2}
\epow dx\\
&=& \frac{1}{\gamma}  \left[
-\,(x-x_0)\,\sigma^2\,\epow +
  A\,{\sqrt{{\frac{\pi }{2}}}}\,
   {{\sigma }^3}\,\erf ({\frac
        {x - x_0} {\sigma {\sqrt{2}}  }})
\right] \biggr|_{-\infty}^{\infty}\\
&=& \frac{2}{\gamma}A\sigma\, \sqrt{\frac{\pi}{2}} \,\,{\sigma}^2 
    = \sigma^2
\end{eqnarray*} 

\begin{equation}
\framebox{$\sigma_0 = \sigma$}
\end{equation}

\vspace{1cm} 
{\bf The Fisher coefficient of Skewness $\xi_1$:}
\begin{eqnarray*}
\xi_1 &=& \frac{1}{\gamma}\int\limits_{-\infty}^{\infty} \left({\frac{x-x_0}{\sigma}}\right)^{3}\epow dx\\
&=& \frac{1}{\gamma} { \epow\,\,
\frac{  -2\,{{\left( x - x_0 \right)}^2}\, - 
       4\,{{\sigma }^2} }{2\,\,
     {{\sigma }}}} \biggr|_{-\infty}^{\infty}\\
&=& 0
\end{eqnarray*}

\begin{equation}
\framebox{$\xi_1 = 0$}
\end{equation} 
This is what we expected for a symmetric distribution like a standard Gauss.

\vspace{1cm} 
{\bf The Fisher coefficient of Kurtosis $\xi_2$:} 
\begin{eqnarray*}
\xi_2 &=& \frac{1}{\gamma}\int\limits_{-\infty}^{\infty} \left({\frac{x-x_0}{\sigma}}\right)^{4}Ae^{\frac{-(x-x_0)^2}{2\sigma^2}}dx\\
&=& \frac{1}{\gamma} \left[ {A\,{e^{{-\frac{{{\left( x - x_0 \right) }^2}}
      {2\,{{\sigma }^2}}}}}\,\,
\frac{  {{-\left( x - x_0 \right)}^3}\, - 
       3\,\left( x - x_0 \right){{\sigma }^2} }{\,
     {{\sigma }^2}}} +
3\,A\,{\sqrt{{\frac{\pi }{2}}}}\,\,
   {\sigma}\,\erf \left({\frac
        {x - x_0} {{\sqrt{2}}\,
        \sigma }}\right) \right]
\Biggr|_{-\infty}^{\infty}\\
&=& 3
\end{eqnarray*}

\begin{equation}
\framebox{$\xi_2 = 3$}
\end{equation} 
\begin{equation}
\framebox{$\xi_f = 0$}
\end{equation} 


\section{Partial derivatives of the standard Gauss function}
The standard Gauss is represented by the formula:
$$\phi(x)=\epow + Z$$
The partial derivatives used in the least squares routine to fit the parameters
are:
\begin{eqnarray}
\frac{\partial\phi(x)}{\partial A} &=& e^{-\frac{1}{2} {\left(\frac{x - x_0}{\sigma}\right)}^2} \\
\frac{\partial\phi(x)}{\partial x_0} &=& \epow \frac{(x-x_0)}{\sigma^2}\\
\frac{\partial\phi(x)}{\partial \sigma} &=& \epow \frac{{(x-x_0)}^2}{\sigma^3} 
\end{eqnarray}

the derivatives for the background $Z = Z_0 + Z_1\, (x-x_0) + Z_2\, (x-x_0)^2$
are:

\begin{eqnarray}
\frac{\partial\phi(x)}{\partial Z_0} &=& 1\\
\frac{\partial\phi(x)}{\partial Z_1} &=& x - x_0\\ 
\frac{\partial\phi(x)}{\partial Z_2} &=& {(x - x_0)}^2\\
\end{eqnarray}



\section{Relation between FWHM and dispersion of a Gauss}

Consider a standard Gauss function without any offset and symmetric around
$x_0 = 0$:
$$f(x) = A{e^{-\frac{1}{2} {(\frac{x}{\sigma})}^2}} $$
To find a width at certain height given the dispersion, 
find a $x_\lambda$ with $0< \lambda < 1$ for which:
 
$$A\,e^{\frac{-x^2}{2\sigma^2}} = \lambda\,A   \longrightarrow
x^2_\lambda = 2 \sigma^2 \ln{\frac{1}{\lambda}}$$

\begin{center}
\framebox {$x_ \lambda = {\sigma \sqrt{2 Ln {\frac {1}{\lambda}}}}$}
\end{center}

A frequently used measure of profile width is the the Full Width at Half Maximum 
$fwhm$. This width can be expressed in $\sigma$ by substitution of
$\lambda = {\frac {1}{2}}$ and multiplication of the solution by 2
because the dispersion is a half width!

\begin{center}

$$fwhm = 2 \sigma\, \sqrt{2 \ln{ \frac{1}{\frac{1}{2}} } }
= 2 \sigma\, \sqrt{2\ln{2}} \approx 2.355\, \sigma  $$


\framebox {$fwhm = 2\sigma\, \sqrt { 2 \ln{2}}$}
\end{center} 
Note that the error in the fitted width at any height, increases or decreases with the same factor.


\chapter{The Gauss-Hermite series}

The Gauss-Hermite $(GH)$ series in {\bf \it XGAUPROF} with 
$y \equiv \frac{x-b}{c}$ is represented by:
$$
\phi(x) = a\,e^{-\frac{1}{2}y^2} \left\{ 1+\frac{h_3}{\sqrt{6}}
(2\sqrt{2}y^3-3\sqrt{2}y) + 
\frac{h_4}{\sqrt{24}}(4y^4-12y^2+3)   \right\} + Z
$$

Simplify this equation further:

$$\phi(x) = a\,E\,\left\{1+h_3(c_1y+c_3y^3)+h_4(c_0+c_2y^2+c_4y^4)\right\}$$
or:
$$\phi(x) =  a\,E\,Q$$
with  $E \equiv e^{-\frac{1}{2}y^2}$,
$Q = \left\{1+h_3(c_1y+c_3y^3)+h_4(c_0+c_2y^2+c_4y^4)\right\}$ and further:

$$c_0 = \frac{1}{4}\sqrt{6}\hspace{0.4cm}
c_1 = -\sqrt{3}\hspace{0.4cm}
c_2 = -\sqrt{6}\hspace{0.4cm}
c_3 = \frac{2}{3}\sqrt{3}\hspace{0.4cm}
c_4 = \frac{1}{3}\sqrt{6}\hspace{0.4cm}
$$

Then the partial derivatives used in the least-squares fit routine are:
\begin{eqnarray}
\frac{\partial\phi(x)}{\partial a} &=& E\,Q\\
\frac{\partial\phi(x)}{\partial b} &=& a\,E\,\frac{1}{c}
\left[
h_3(-c_1-3c_3y^2)+h_4(-2c_2y-4c_4y^3)+y\,Q
\right]\\
\frac{\partial\phi(x)}{\partial c} &=& a\,E\,\frac{1}{c}
\left[
h_3(-c_1y-3c_3y^3)+h_4(-2c_2y^2-4c_4y^4)+y^2\,Q 
\right] = 
y\,\frac{\partial\phi(x)}{\partial b}\\
\frac{\partial\phi(x)}{\partial h_3} &=& a\,E\,(c_1y+c_3y^3)\\
\frac{\partial\phi(x)}{\partial h_4} &=& a\,E\,(c_0+c_2y^2+c_4y^4)
\end{eqnarray}


Note that $\frac{\partial\phi(x)}{\partial x} = -\frac{\partial\phi(x)}{\partial b}$.
We will need this expression if we want to calculate the position of
the maximum of the $GH$ series. This position is only equal to parameter $b$
if $h_3=h_4=0$!
To find the real maximum, solve:

$$
\frac{\partial\phi(x)}{\partial x} = -a\,E\,\frac{1}{c}
\left[
h_3(-c_1-3c_3y^2)+h_4(-2c_2y-4c_4y^3)+y\,Q
\right] = 0
$$
which means solving:
$$
h_3(-c_1-3c_3y^2)+h_4(-2c_2y-4c_4y^3)+y\,
\left\{1+h_3(c_1y+c_3y^3)+h_4(c_0+c_2y^2+c_4y^4)\right\} = 0
$$
After rearranging the equation above we get:
\begin{equation}
\lambda_5y^5+\lambda_4y^4+\lambda_3y^3+\lambda_2y^2+\lambda_1y+\lambda_0=0
\end{equation}
with coefficients equal to:
\begin{eqnarray*}
\lambda_0 &=&-h_3\,c_1\\
\lambda_1 &=& h_4\,(c_0-2c_2)+1\\
\lambda_2 &=& h_3\,(c_1-3c_3)\\
\lambda_3 &=& h_4\,(c_2-4c_4)\\
\lambda_4 &=& h_3\,c_3\\
\lambda_5 &=& h_4\,c_4
\end{eqnarray*}
We applied a bisection to solve the equation. For initial limits we use
$x_1 = b-\frac{1}{2}c$ and $x_2 = b+\frac{1}{2}c$ which for the 
bisection equation is the same as 
$y_1 = -\frac{1}{2}$ and $y_2 = \frac{1}{2}$.


\section{Moments of the $GH$ series}
As stated before, in the function:
$$\phi(x) = a\,e^{-\frac{1}{2}y^2},\left\{1+h_3(c_1y+c_3y^3)+h_4(c_0+c_2y^2+c_4y^4)\right\}$$
with:
$$y \equiv \frac{x-b}{c}$$
the parameters $a$, $b$ and $c$  are only equal to $A$, $x_0$ and $\sigma$ of
a standard Gaussian, if $h_3=h_4=0$. If we want to compare these parameters 
for $h_3$ and/or $h_4$ unequal to $0$ then we have to calculate the  profile properties using the moments of the
GH distribution:

\medskip
{\bf The integrated line strength $\gamma $:}
$$
\gamma = \int\limits_{-\infty}^{\infty} \phi(x) dx =
 a\,c\, \sqrt{2\pi} \ampfac
= a\,c\, \sqrt{2\pi} (1+\frac{1}{4}\sqrt{6}\,h_4)
$$
\begin{equation}
\framebox{$\gamma = a\,c\, \sqrt{2\pi} (1+\frac{1}{4}\sqrt{6}\,h_4)$}
\end{equation}

\vspace{1cm}
{\bf The mean abscissa $X_0$:}
\begin{eqnarray*}
X_0 &=& \frac{1}{\gamma}\int\limits_{-\infty}^{\infty} x \phi(x) dx\\
&=& \frac{1}{\gamma}  a\,c\, \sqrt{2\pi}\ 
(b+c\, c_1\,h_3+3c\,c_3\,h_3+b\,c_0\,h_4+b\,c_2\,h_4+3b\,c_4\,h_4)\\
&=& \frac{1}{\gamma}  a\,c\, \sqrt{2\pi}
\left[
b\, (1+\frac{1}{4}\sqrt{6}\,h_4)+c\, (\sqrt{3}\,h_3)
\right]\\
&=& b + c\,\frac{\sqrt{3}\,h_3}{1+\frac{1}{4}\sqrt{6}\,h_4}\\
&\approx & b + c\,\left[\sqrt{3}\,h_3\,(1-\frac{1}{4}\sqrt{6}\,h_4)\right]
\end{eqnarray*} 

The last step is an approximation. Remember that 
${(1+\alpha\,x)}^\beta \approx 1-\alpha\,\beta\,x$ for small $x$.
If we include only the lowest order terms in $h_3$ and $h_4$, then:

\begin{equation} 
\framebox{$X_0 \approx b + \sqrt{3}\,h_3\,c$}
\end{equation}

\vspace{1cm}
{\bf The dispersion $\sigma_0$:}
\begin{eqnarray*}
\sigma^2_0 &=& \frac{1}{\gamma}\int\limits_{-\infty}^{\infty} (x-X_0)^{2}
\phi(x) dx\\
&\approx & \frac{1}{\gamma}\,a\,c\, \sqrt{2\pi}\, c^2\,\, 
\frac{
\left( 
1+3c_0\,h_4+5c_2\,h_4+21c_4\,h4
\right)
}
{ \ampfac^2 } \\ 
&=& c^2\, \frac{\left(1+3c_0\,h_4+5c_2\,h_4+21c_4\,h4 \right) }
{\ampfac^3 }\\
&=& c^2\,
\frac{1+2\frac{3}{4}\sqrt{6}\,h_4}{{\left(1+\frac{1}{4}\sqrt{6}\,h_4\right)}^3}\\
&\approx & c^2\,(1+2\sqrt{6}\,h_4)
\end{eqnarray*} 

\begin{equation}
\framebox{$\sigma_0 \approx c\sqrt{1+2\sqrt{6}\,h_4} \approx c\,(1+\sqrt{6}\,h_4)$}
\end{equation}

\vspace{1cm}
{\bf The Fisher coefficient of Skewness $\xi_1$:}\\
A set of observations that is not symmetrically distributed is said to be skewed.
If the distribution has a longer tail less than the maximum,
the function has {\it negative skewness}. Otherwise, it has
{\it positive skewness}.
\begin{eqnarray*}
\xi_1 &=& \frac{1}{\gamma}\int\limits_{-\infty}^{\infty} \left({\frac{x-x_0}{\sigma}}\right)^{3} \phi(x) dx\\
&\approx & \frac{1}{\gamma}\,\,\frac{ 2a\,c\, \sqrt{2\pi}\,c^3\,3\,c_3h_3}
{\ampfac^3 {\left(\frac{c^2\left(1+h_4\,(3c_0+5c_2+21c_4)\right)}{\ampfac^3} \right)}^\frac{3}{2} }\\ 
&=& \frac{6\,c_3\,h_3\, \ampfac^\frac{1}{2} }
{{\left(1+h_4\,(3c_0+5c_2+21c_4)\right)}^\frac{3}{2}}\\
&=& \frac{4\sqrt{3}\,h_3\, {(1+\frac{1}{4}\sqrt{6}\,h_4)}^\frac{1}{2}}
{{\left(1+2\frac{3}{4}\sqrt{6}\,h_4\right)}^\frac{3}{2}}
\end{eqnarray*} 

\begin{equation}
\framebox{$\xi_1 \approx 4\sqrt{3}\,h_3$}
\end{equation} 

This is what we could have expected because $h_3$ is the parameter that
measures asymmetric deviations.

\vspace{1cm}
{\bf The Fisher coefficient of Kurtosis $\xi_2$:}
\begin{eqnarray*}
\xi_2 &=& \frac{1}{\gamma}\int\limits_{-\infty}^{\infty} \left({\frac{x-X_0}{\sigma}}\right)^{4} \phi(x) dx\\
&\approx & \frac{1}{\gamma}\,\, \frac{-3ac\,\sqrt{2\pi}\,{\left(1+c_0h_4+c_2h_4+3c_4h_4 \right)}^2
 \left(-1-5c_0h_4-9c_2h_4-47c_4h_4\right)}
{{\left(1+3c_0h_4+5c_2h_4+21c_4h_4 \right)}^2}\\
&=& \frac{3\,(1+h_4\,(c_0+c_2+3c_4)(1+h_4\,(5c_0+9c_2+47c_4)}
{{\left(1+h_4\,(3c_0+5c_2+21c_4)\right)}^2}\\
&=& \frac{3(1+\frac{1}{4}\sqrt{6}\,h_4)(1+\frac{95}{12}\sqrt{6}\,h_4)}
{{\left(1+2\frac{3}{4}\sqrt{6}\,h_4\right)}^2}
\end{eqnarray*}

\begin{equation}
\framebox{$\xi_2 \approx 3 + 8\sqrt{6}\,h_4$}
\end{equation} 
\begin{equation}
\framebox{$\xi_f \approx 8\sqrt{6}\,h_4$}
\end{equation} 

The initial guesses for the $h_3$ and $h_4$ parameters in the least-squares
fit, are set to zero because we expect that the profile will still resemble
the standard Gaussian. If a fit is successful,
the profile parameters $\gamma$, $X_0$ and $\sigma$, skewness and
kurtosis are calculated from
$a$,$b$, $c$, $h_3$ and $h_4$ using the formulas above.
For the errors in these parameters we derived:
\begin{eqnarray}
\Delta \gamma &=& \frac{1}{\gamma}\, 
\sqrt{{\left(\frac{\Delta a}{a}\right)}^2+
      {\left(\frac{\Delta c}{c}\right)}^2+ 
      {\left(\frac{1}{\frac{2}{3}\sqrt{6}+h_4}\right) }^2 
      {\left(\frac{\Delta h_4}{h_4}\right)}^2 }\\
\Delta X_0 &=& 
\sqrt{ {(\Delta b)}^2 + 3h_3^2{(\Delta c)}^2 +
       3c^2 {(\Delta h_3)}^2 }\\
\Delta \sigma_0 &=& 
\sqrt{{(1+\sqrt{6}\,h_4)}^2\, {(\Delta c)}^2 + 6c^2{(\Delta h_4)}^2  }\\
\Delta \xi_1 &=&  4\sqrt{3}\,\Delta h_3\\
\Delta \xi_2 &=&  8\sqrt{6}\,\Delta h_4
\end{eqnarray}



\chapter{The Voigt profile} 

The Voigt profile is a line-shape which results form a convolution of Lorentzian and Doppler line broadening mechanisms:
\begin{equation}
 \phi_{Lorentz}(\nu)=\frac{1}{\pi} \frac{\alpha_L}{(\nu-\nu_0)^2 + \alpha_L^2}
\end{equation} 
\begin{equation} 
 \phi_{Doppler}(\nu)=\frac{1}{\alpha_D} \sqrt{\frac{\ln{2}}{\pi}} e^{-\ln{2} \frac{(\nu-\nu_0)^2}{\alpha_D^2}}
\end{equation}

Both functions are normalized. {\bf $\alpha_D$ and $\alpha_L$ are half widths
at half maximum}.
Convolution is given by the relation 
$$ f(\nu) \star g(\nu)=\int\limits_{-\infty}^\infty {f(\nu - t ) g(t) dt} $$\\
Define the ratio of Lorentz to Doppler widths as 

\begin{equation}
 \label{eq:y}
 y \equiv \frac{\alpha_L}{\alpha_D} \sqrt{\ln{2}}
\end{equation} 

and the frequency scale (in units of the Doppler Line-shape half-width  $\alpha_D$ )
 
\begin{equation}
 \label{eq:x}
 x \equiv \frac{\nu-\nu_0}{\alpha_D} \sqrt{\ln{2}} 
\end{equation}
then:
\[ \phi_L(\nu) = \frac{1}{\pi} \frac{\sqrt{\ln{2}}}{\alpha_D} \frac{y}{x^2+y^2} \]
and
\[ \phi_D(\nu) = \frac{1}{\alpha_D} \sqrt{\frac{\ln{2}}{\pi}} e^{-x^2} \]
The convolution:
\[ \phi_L(\nu)\star\phi_D(\nu)=\int\limits_{-\infty}^\infty {\frac{1}{\pi} \frac{\sqrt{\ln{2}}}{\alpha_D} \frac{y}{(x-t')^2 + y^2} \frac{1}{\alpha_D} \frac{\sqrt{\ln{2}}}{\sqrt{\pi}} e^{-t'^2} dt}\]

If you replace $\nu$ by $\nu-t$  in the expression for $x$, then 
\[ x-t'=\frac{\nu-t}{\alpha_D}\sqrt{\ln{2}}-\frac{\nu_0}{\alpha_D}\sqrt{\ln{2}}=
\frac{\nu-\nu_0}{\alpha_D}\sqrt{\ln{2}}-t\frac{\sqrt{\ln{2}}}{\alpha_D}=
x-\frac{\sqrt{\ln{2}}}{\alpha_D}t \]
and we conclude that $ t'=\frac{\sqrt{\ln{2}}}{\alpha_D}t $ and
$ dt'=\frac{\sqrt{\ln{2}}}{\alpha_D}dt 
\Leftrightarrow dt=\frac{\alpha_D}{\sqrt{\ln{2}}}dt' $

\[ \phi_L(\nu)\star\phi_D(\nu)=\frac{1}{\alpha_D}\frac{\sqrt{\ln{2}}}{\alpha_D}
\frac{\sqrt{\ln{2}}}{\sqrt{\pi}}\frac{\alpha_D}{\sqrt{\ln{2}}}\, \,
\frac{y}{\pi}\int\limits_{-\infty}^\infty {\frac{e^{-t'^2}}{(x-t')^2+y^2}dt'} \]

Replace $t'$ by $t$ to obtain:

\begin{equation} 
\phi_\nu(\nu)=\phi_L(\nu)\star\phi_D(\nu)=
\frac{1}{\alpha_D}\sqrt{\frac{\ln{2}}{\pi}}\, \frac{y}{\pi}\int\limits_{-\infty}^\infty {\frac{e^{-t^2}}{(x-t)^2+y^2}dt}\,
\rightarrow
\label{Voigtprofile}
\end{equation} 


\begin{equation}
\phi_\nu(\nu)=
\frac{1}{\alpha_D}\sqrt{\frac{\ln{2}}{\pi}}\,
\frac{1}{\pi}\,
\frac{\alpha_L}{\alpha_D} \sqrt{\ln{2}}\,
\int\limits_{-\infty}^\infty {\frac{e^{-t^2}}{\left(\frac{\nu-\nu_0}{\alpha_D} \sqrt{\ln{2}}-t\right)^2+\,
\left({\frac{\alpha_L}{\alpha_D} \sqrt{\ln{2}}}\right)^2}dt}\,
\rightarrow
\label{Voigtprofile_ex1}
\end{equation} 

\fbox{
 \addtolength{\linewidth}{-2\fboxsep}%
 \addtolength{\linewidth}{-2\fboxrule}%
 \begin{minipage}{\linewidth}

\begin{equation}
\phi_\nu(\nu)=
\frac{\alpha_L}{\alpha_D^2}
\frac{\ln{2}} {\pi^{\frac{3}{2}}}
\int\limits_{-\infty}^\infty {\frac{e^{-t^2}}{\left(\frac{\nu-\nu_0}{\alpha_D} \sqrt{\ln{2}}-t\right)^2+\,
\left({\frac{\alpha_L}{\alpha_D} \sqrt{\ln{2}}}\right)^2}dt}
\label{Voigtprofile_ex2}
\end{equation}

 \end{minipage}
}

\vspace{5 mm}
Note that $\alpha_L$ and $\alpha_D$ are both {\bf half-width} and not FWHM's.
As we will see in the next section: 
$ \int\limits_{-\infty}^\infty {\phi_\nu(\nu)d\nu} = 1 $
so the Voigt profile (eq.\ref{Voigtprofile} or eq.\ref{Voigtprofile_ex2}) is also normalized.



\section{The Voigt function $K(x,y)$} 

Part of the expression for the Voigt line-shape is the Voigt function.
The formula for this function is:

\[
K(x,y) = \frac{y}{\pi} {\int\limits_{- \infty} ^{\infty}} \frac{e^{-t^{2}}}{y^2 + {(x - t)}^2} dt
\]

Rewrite the Voigt function using the integral:

\begin{displaymath}
\int\limits_{0}^{\infty}
e^{-yv}\cos{\left( (x-t)v\right)}dv = 
\frac{y}{(x-t)^2+y^2}
\end{displaymath}

Then:

\begin{eqnarray*} 
K(x,y) &=& 
\frac{1}{\pi}
\int\limits_{-{\infty}}^{+{\infty}}
\frac{ye^{-t^2}}{(x-t)^2+y^2}\,dt\\
&=&
\frac{1}{\pi}
\int\limits_{-{\infty}}^{+{\infty}}
e^{-t^2}
\int\limits_{0}^{\infty}
e^{-yv}\cos{\left((x-t)v \right)}\,dv\,dt\\
&=&
\frac{1}{\pi}
\int\limits_{0}^{\infty}
e^{-yv}
\int\limits_{-{\infty}}^{+{\infty}}
e^{-t^2}\cos{\left( (x-t)v\right)}\, dt\,dv\\
&=&
\frac{1}{\pi}
\int\limits_{0}^{\infty}
e^{-yv} \cos{xv}
\int\limits_{-{\infty}}^{+{\infty}}
e^{-t^2}\cos{tv}\, dt\,dv
\end{eqnarray*}  

where we used the relation:

\begin{displaymath}
\cos(a-b) = \cos a\cos b + \sin a \sin b
\end{displaymath}

and the fact that the sine part of the intergral evaluate to zero.

To continue, we use another well known integral:
 

\begin{displaymath}
\int\limits_{-{\infty}}^{+{\infty}}
e^{-t^2}\cos{(tv)}dt =
\sqrt{{\pi}}e^{-{\frac{v^2}{4}}} \rightarrow
\end{displaymath}

\begin{eqnarray*} 
K(x,y) &=& 
\frac{1}{\sqrt{\pi}}
\int\limits_{0}^{\infty}
e^{-{\frac{v^2}{4}}}e^{-yv}\cos(xv)\,dv\\
&=&
\frac{1}{2{\sqrt{\pi}}}
\int\limits_{0}^{\infty}
e^{-{\frac{v^2}{4}}-yv}
(e^{ixv}+e^{-ixv})dv \\
&=&
\frac{1}{2{\sqrt{\pi}}}
\int\limits_{0}^{\infty}
(e^{-{\frac{v^2}{4}}-(y-ix)v}+e^{-{\frac{v^2}{4}}-(y+ix)v})dv
\end{eqnarray*}  

Let $z = x + iy$, then $iz = -(y-ix)$, i${\overline{z}} = (y +ix)$
then the Voigt function is:

\begin{displaymath}
K(x,y) = 
\frac{1}{2{\sqrt{\pi}}}
\int\limits_{0}^{\infty}
(e^{-({\frac{v}{2}}-iz)^2-z^2}+
e^{-({\frac{v}{2}}+i{\overline{z}})^2-{\overline{z}}^2})dv
\end{displaymath}

with:

\begin{displaymath}
\framebox{$z = x + iy$}
\end{displaymath}

and $x$ and $y$ from equations \ref{eq:x} and \ref{eq:y}.


Let $u1=v/2 - iz$ and $u_2 = v/2 + iz$:

\[
K(x,y) = \frac{1}{\sqrt{\pi}} \ e^{-z^2} {\int\limits _{- iz} ^{\infty}} {e^{-{u_1}^{2}}} d{u_1} \ + \ \frac{1}{\sqrt{\pi}} {e^{- \bar{z} ^2}} {\int\limits _{i \bar{z}} ^{\infty}} {e^{-{u_2}^{2}}} d{u_2}
\]

According to Abramowitz \& Stegun (AS): 
$\erfc (z) \equiv \frac{2}{\sqrt{\pi}} {\int\limits _{z} ^{\infty}} {e^{-t^{2}}} dt $
so that: 

\begin{equation}  
K(x,y) = \frac{1}{2} \ \left({e^{-z^{2}}}  \erfc (-iz) \ + {e^{- \bar{z}^{2}}} \erfc (i \bar{z})\right)\
\end{equation}

In AS we find the function $\omega(z)$ defined as: 
\begin{equation}
\omega(z) = e^{-z^2} \erfc (-iz)
\end{equation}

With the relations above one recognizes 
the relation $ \omega( \bar{z}) \ = \ \overline{\omega(-z)} $.
If we write the Voigt function in terms of function $w$ then:


$$
 K(x,y)  = \frac{1}{2} \{ \omega(z) + \omega(- \bar{z} ) \} = 
\frac{1}{2} \{ \omega(z) + \overline{\omega(z)} \} = \Re \{ \omega(z) \}
$$
\begin{equation}
\framebox{$K(x,y) = \Re \{\omega(z)\}$ } 
\end{equation}

\newpage



\section {Area under the Voigt Line-shape}
Remember the Voigt line-shape  was given by:\\
\begin{equation}
\phi (\nu) = \frac {A} {\alpha_D} \sqrt{\frac {\ln 2} {\pi}} \frac {y} {\pi} \int\limits_{t=-\infty}^{t=\infty} \frac {e^{-t^2}} {y^2+(x-t)^2} dt \end{equation}\\

We want to evaluate 
$\int \limits_{-\infty}^{\infty} \phi (\nu) d\nu.$
By changing the integration order, one can write:
$$
\int\limits_{\nu=-\infty}^{\nu=\infty} \phi (\nu) d\nu = \frac {A} {\alpha_D} \sqrt{\frac {\ln 2} {\pi}} \int\limits_{t=-\infty}^{t=\infty} \frac {y} {\pi} \left \{\int\limits_{\nu=-\infty}^{\nu=\infty} \frac {e^{-t^2}} {y^2+(x-t)^2} d\nu \right \} dt
$$
Because $x \equiv \frac {\nu - \nu_0} {\alpha_D} \sqrt{\ln 2}$, we derive $d\nu = \frac {\alpha_D} {\sqrt{\ln2}} dx$
$$
\int\limits_{-\infty}^{\infty} \phi (\nu) d\nu = \frac {A} {\alpha_D} 
\sqrt{\frac {\ln 2} {\pi}}  \frac {\alpha_D} {\sqrt{\ln2}} 
\int\limits_{-\infty}^{\infty} \frac {y} {\pi} 
\left \{ \int\limits_{-\infty}^{\infty} \frac {e^{-t^2}} {y
^2+(x-t)^2} dx \right \} dt
$$
The inner integral $\int\limits_{-\infty}^{\infty} \frac {1} {y^2+(x-t)^2} dx = \frac {1} {y} \arctan \Bigl(\frac {x-t} {y} \Bigr) \Bigr |_{-\infty}^{\infty} = \frac {\pi} {y}$
so that: 
$$
\int\limits_{-\infty}^{\infty} \phi (\nu) d\nu = \frac {A} {\sqrt{\pi}} \int\limits_{-\infty}^{\infty} e^{-t^2} dt 
= {\frac {A} {\sqrt{\pi}} \cdot \sqrt{\pi}}  =  A
$$

This proves what we already expected. Besides the area scaling factor $A$ 
(which is a fit parameter in the least-squares fit) 
the area of the convolution of the two functions 
(Lorentzian and Doppler line-shapes) is normalized.
The amplitude is found at $\nu=\nu_0$. Then according to its definition
$x=0$ and the relation between the amplitude and area is 
$amp=\phi (\nu_0)$:


\begin{equation}
\framebox {$
amp = \frac {A} {\alpha_D} \sqrt{\frac {\ln 2} {\pi}} K(0,y) 
$}
\end{equation}
The error in the amplitude is then:


\begin{eqnarray*}
\Delta amp &=& \sqrt{\frac {\ln 2}{\pi}} \, K(0,y)\,  
\sqrt{\Bigl(\frac {\partial} {\partial\alpha_D} 
\frac {A} {\alpha_D} \Delta \alpha_D \Bigr)^2 + 
\Bigl(\frac {\partial} {\partial A} \frac {A} {\alpha_D} \Delta A \Bigr)^2}\\
&=& \sqrt{\frac {\ln 2} {\pi}} \, K(0,y)\,
\sqrt{  { \left( \frac{A}{\alpha_D ^2}\Delta\alpha_D^2\right)}^2+
        {\left(\frac{1}{\alpha_D ^2}\Delta A \right)}^2 }
\end{eqnarray*} 

\begin{equation}
\framebox {$ \Delta amp =
amp\, \sqrt{ {\left(\frac{\Delta\alpha_D}{\alpha_D}\right)}^2 + 
                {\left(\frac{\Delta A}{A} \right) }^2  }
$}
\end{equation}


\section{Partial derivatives of the Voigt Function}
Consider the relations $z=x+iy$ and $-\bar{z}=-x+iy$. Then:
\[\frac{\partial\omega}{\partial x}=\frac{\partial z}{\partial x}\frac{\partial\omega}{\partial z}=\frac{\partial(x+iy)}{\partial x}\frac{\partial\omega}{\partial z}=\frac{\partial\omega}{\partial z}\]
\[\frac{\partial\omega}{\partial x}=\frac{\partial(-\bar{z})}{\partial x}\frac{\partial\omega}{\partial(-\bar{z})}=\frac{\partial(-x+iy)}{\partial x}\frac{\partial\omega}{\partial(-\bar{z})}=\frac{-\partial\omega}{\partial(-\bar{z})}\]
\[\frac{\partial\omega}{\partial y}=\frac{\partial z}{\partial y}\frac{\partial\omega}{\partial z}=\frac{\partial(x+iy)}{\partial y}\frac{\partial\omega}{\partial z}=i\frac{\partial\omega}{\partial z}\]
\[\frac{\partial\omega}{\partial y}=\frac{\partial(-\bar{z})}{\partial y}\frac{\partial\omega}{\partial(-\bar{z})}=\frac{\partial(-x+iy)}{\partial y}\frac{\partial\omega}{\partial(-\bar{z})}=i\frac{-\partial\omega}{\partial(-\bar{z})}\]
Now it is necessary to find $\frac{\partial\omega(z)}{\partial z}$. In Abramowitz \& Stegun we find the relation $\frac{\partial\omega(z)}{\partial z}=-2z\omega(z)+\frac{2i}{\sqrt{\pi}}$. Then:
\begin{eqnarray*}
\frac{\partial K(x,y)}{\partial x}&=&\frac{\partial}{\partial x}\left\{ \frac{1}{2} \left( \omega(z)+\omega(-\bar{z})\right)\right\}\\
&=&\frac{1}{2}\left\{\frac{\partial\omega(z)}{\partial x}+\frac{\partial\omega(-\bar{z})}{\partial x}\right\}\\
&=&\frac{1}{2}\left\{\frac{\partial}{\partial z}\omega(z)-\frac{\partial}{\partial(-\bar{z})}\omega(-\bar{z})\right\}\\
&=&\frac{1}{2}\left\{-2z\omega(z)+\frac{2i}{\sqrt{\pi}}-\left(-2(-\bar{z})\omega(-\bar{z})+\frac{2i}{\sqrt{\pi}}\right)\right\}\\
&=&-\left(z\omega(z)+\bar{z}\omega(-\bar{z})\right)\\
\mbox{with }\omega(-\bar{z})=\overline{\omega(z)}\mbox{:}&&\\
&=&-\left(z\omega(z)+\overline{z\omega(z)}\right)\\
&=&-2\Re\left\{z\omega(z)\right\} 
\end{eqnarray*}



In the same way we write:

\begin{eqnarray*}
\frac{\partial K(x,y)}{\partial y} 
&=& \frac{\partial}{\partial y} \left\{ \frac{1}{2}
\left( \omega(z)+\omega(-\bar{z})\right) \right\}\\
&=&\frac{1}{2}\left\{\frac{\partial\omega(z)}{\partial y}+\frac{\partial\omega(-\bar{z})}{\partial y}\right\}\\
&=& {i\over 2}\left\{ {\partial \omega(z)\over \partial z} + i{\partial \omega(-\overline{z})\over \partial (-\overline{z})}\right\} \\
&=& {i\over 2}\left\{-2z\omega(z) + {2i\over \sqrt{\pi}} +  -z(-\overline{z})\omega(-\overline{z}) + {2i\over \sqrt{\pi}}\right\}\\
&=& -{2\over \sqrt{\pi}} -i\left\{z\omega(z) - \overline{z}\omega(-\overline{z})\right\}\\
&=& -{2\over \sqrt{\pi}} -i\{z\omega(z) - \overline{z\omega(z)}\} 
\hspace{1cm} (because\,\, \omega(-\overline{z}) = \overline{\omega(z)})\\
&=& -{2\over \sqrt{\pi}} + 2 \Im\{z\omega(z)\}
\end{eqnarray*}

To summarize:
\begin{equation}
\framebox {$ {\partial K(x,y)\over \partial y} = -2 \Re\{z\omega(z)\} $}
\end{equation}
\begin{equation}
\framebox {${\partial k(x,y)\over \partial y} = -{2\over \sqrt{\pi}} + 2 \Im\{z\omega(z)\}  $}
\end{equation}


\section{Partial derivatives of the Voigt Line-shape}
Lets start to recall some definitions:
$$
\phi \left( v \right)={A \over {\alpha_D}}\sqrt {{{\ln{2}} \over
\pi }}K\left( {x,y} \right) \hspace{1cm}
K\left( {x,y} \right)=\Re \left\{ {\omega \left( z
\right)} \right\}
$$
with:
$$
x={{v-v_0} \over {\alpha_D}}\sqrt {\ln{2}} \hspace{1cm}
y={{\alpha_L} \over {\alpha_D}}\sqrt {\ln{2}}
$$
$$
  {\partial  \over {\partial x}}K\left( {x,y} \right)=-2\Re \left\{ {z\omega
\left( z \right)} \right\},\,\, {\partial  \over {\partial y}}K\left( {x,y}
\right)=-{2 \over {\sqrt \pi }}+2\Im \left\{ {z\omega \left( z \right)}
\right\}
$$

Then:
\begin{eqnarray}
{{\partial \phi } \over {\partial A}} &=& {1 \over {\alpha_D}}\sqrt {{{\ln{2}}
\over \pi }}K\left( {x,y} \right)={1 \over {\alpha_D}}\sqrt {{{\ln{2}} \over
\pi }}\,\Re \left\{ {\omega \left( z \right)} \right\}\\
{{\partial \phi } \over {\partial v_0}} &=& {A \over {\alpha_D}}\sqrt {{{\ln{2}}
\over \pi }}{\partial  \over {\partial x}}K\left( {x,y} \right){{\partial x}
\over {\partial v}}={A \over {\alpha_D}}\sqrt {{{\ln{2}} \over
\pi }}\biggl\{ -2\Re \left\{ {z\omega \left( z \right)} \right\}\biggr\}{{-1} \over
{\alpha_D}}\sqrt {\ln{2}}\\
{{\partial \phi } \over {\partial \alpha_L}} &=& {A \over {\alpha_D}}\sqrt
{{{\ln{2}} \over \pi }}{\partial  \over {\partial y}}K\left( {x,y}
\right){{\partial y} \over {\partial v}}={A \over {\alpha_D}}\sqrt {{{\ln{2}}
\over \pi }}\left\{ {-{2 \over {\sqrt \pi }}+2\Im \left\{ {z\omega \left( z
\right)} \right\}} \right\}{1 \over {\alpha_D}}\sqrt {\ln{2}}\\
{{\partial \phi } \over {\partial \alpha_D}} & = & K\left( {x,y}
\right){\partial  \over {\partial \alpha_D}}\left\{ {{A \over {\alpha
_D}}\sqrt {{{\ln{2}} \over \pi }}} \right\}+{A \over {\alpha_D}}\sqrt {{{\ln{2}}
\over \pi }}{\partial  \over {\partial \alpha_D}}K\left( {x,y} \right) \nonumber\\
& = & A\sqrt {{{\ln{2}} \over \pi }}\left( {-{{K\left( {x,y} \right)} \over
{\alpha_D^2}}+{1 \over {\alpha_D}}\left\{ { {\partial \over {\partial
x}}K\left( {x,y} \right){\partial  \over {\partial \alpha_D
}}\left[ {{{v-v_0} \over {\alpha_D}}\sqrt {\ln{2}}} \right]+
{\partial \over
{\partial y}}K\left( {x,y} \right){\partial  \over {\partial \alpha
_D}}\left[ {{{\alpha _L} \over {\alpha_D}}\sqrt {\ln{2}}} \right]} \right\}}
\right) \nonumber\\
& = & A\sqrt {{{\ln{2}} \over \pi }}\left( {-{{K\left( {x,y} \right)} \over
{\alpha_D^2}}+{1 \over {\alpha_D}}\left\{ {-{{v-v_0} \over {\alpha
_D^2}}\sqrt {\ln{2}}{\partial  \over {\partial x}}K\left( {x,y}
\right)-{{\alpha_L} \over {\alpha_D^2}}\sqrt {\ln{2}}{\partial  \over
{\partial y}}K\left( {x,y} \right)} \right\}} \right) \nonumber\\ 
& = & {A \over {\alpha_D}}\sqrt {{{\ln{2}} \over \pi }}{1 \over {\alpha
_D}}\left( {-K\left( {x,y} \right)+\left\{ {-x{\partial  \over {\partial
x}}K\left( {x,y} \right)-y{\partial  \over {\partial y}}K\left( {x,y}
\right)} \right\}} \right)\nonumber\\ 
& = &  -{A \over {\alpha_D}}\sqrt {{{\ln{2}} \over \pi }}{1 \over {\alpha
_D}}\left( {K\left( {x,y} \right)+x{\partial  \over {\partial x
}}K\left( {x,y} \right)+y{\partial  \over {\partial y}}K\left( {x,y}\right)} \right) 
\end{eqnarray}


\chapter{Initial estimates}

We automated calculating initial estimates using a method described by 
Schwarz (1968) called the {\it gauest} method. A second-order polynomial is 
fitted at each pixel  position $x_k$ using $2Q+1$ points distributed 
symmetrically around $x_k$. 
The value at $x_k$ is {\it not} the profile value at that point, but
the median of that value and the two neighbours. Outliers that are not 
filtered with this median filter can be masked in the plot with the 
left mouse button. 
In $XGAUPROF$ the factor $Q$ is called the smoothing factor.
The coefficient of the second-order term of the polynomial is an
approximation of the second derivative of the observed profile. Assuming
that the observed profile can be approximated by the sum of a few (standard)
Gaussian functions, the parameters $x_0$ and $\sigma_0$ are calculated
from the main minima of the second derivative. The amplitude is derived
from the observed profile. Tests using different threshold values make
it possible to discriminate against spurious components; the threshold values 
depend on Q and the r.m.s. noise of the observed profile. The Gaussian
components are then subtracted from the observed profile, and the residual 
profile, mainly the sum of a few remaining broader components, is handled 
in the same way as the original.
Profiles with a maximum amplitude smaller than a user given value and
profiles with a dispersion smaller than a user given value are discarded.
The initial values for these filters are set to zero.
Blanks in the profile and points that are masked by the user are set to 
the value of the zero level. If you did not fix a certain zero level, then
the median of the profile is used as an estimate. If you did not fix
the r.m.s. noise of the observed profile, the distance between the quartiles
of the profile is used as an estimate.
\chapter{References}


\item Abramowitz, M. and Stegun, C. A. (Eds.). {\it Handbook of Mathematical 
Functions with Formulas, Graphs, and Mathematical Tables}, 9th printing. New York: Dover, p. 928, 1972. 
\item Armstrong, B.H., {\it Spectrum Line Profiles: The Voigt Function}. J. Quant. Spectrosc. Radiat. Transfer {\bf 7}, 61-88, 1967\\
\item Kaper, H.G., Smits, D.W., Schwarz, U.J., Takabuko, K., Woerden, H. van,
{Computer Analysis of Observed Distributions into Gaussian components},
Bull. Astr. Inst. Netherlands, {\bf 18} 465-487, 1966 
\item Marel, P. van der, Franx, M.,\,\,{\it A new method for the identification of non-gaussian
line profiles in elliptical galaxies}. A.J., {\bf 407} 525-539, 1993 April 20\\
\item Schwarz, U.J., {\it Analysis of an observed function into components
using its second derivative}, Bull. Astr. Inst. Netherlands, {\bf 19} 405-413, 1968
\end{flushleft}
\end{document}



