\documentstyle[12pt]{report}
\def\H{{Hermes}}
\def\tH{{tHermes}}
\def\xH{{xHermes}}
\def\nH{{nHermes}}
\def\G{{\sc gipsy}}
\def\COA{{\sc coa}}
\def\UCA{{\sc uca}}
\def\TSA{{\sc tsa}}
\parskip=3mm
\parindent=0mm
\makeindex
\evensidemargin=0cm
\oddsidemargin=0cm
\textwidth=16cm
\textheight=21cm
\begin{document}
%\pagestyle{heading}
\thispagestyle{empty}
\begin{titlepage}
\null
\begin{center}
{\bf\huge Hermes User's Guide\\}
\vskip 2.0 truecm
{\large by J.~P.~Terlouw\\}
\vskip 1.0 truecm
Version of\\
April $10^{\rm th}$, 2006
\end{center}
%\vskip 10.0truecm
\vfill
\begin{quote}
\it
\hspace{3mm}``Hermes was de god, die de wegen en het verkeer daarlangs be\-scherm\-de.
Hij was dan ook de god van de reizigers en de handel, zelfs van de dieven, die
immers ook hun geluk langs de wegen zoeken.
Hij was de bode der goden, toegerust met vleugelschoenen en
heraut\-staf, die op
bevel van Zeus boodschappen overbracht en mensen veilig naar hun doel voerde;
ook geleidde hij de doden naar de onder\-wereld.''
\begin{flushright}
\rm
Uit: Ernst Hoffmann:\\
``Goden- en Heldensagen''\\
bewerkt door Dr.~J.~H.~Croon\\
\end{flushright}
\end{quote}
\vfill
This guide describes the user interface of the
Groningen Image Processing SYstem (\hbox{\G}).
It is meant as both an introduction for new users and
a reference document for more experienced users.

\end{titlepage}
\thispagestyle{empty}\strut\newpage
\thispagestyle{empty}
\pagenumbering{roman}
\tableofcontents
\newpage
\pagenumbering{arabic}
\chapter{Introduction}
\section*{What is \H}
\H\ is the user interface of \G.
It enables users to do multi-tasking in an organised fashion.
Users have direct control over the application programs ({\em tasks\/}) they run:
a running task can be suspended or aborted with a simple command; various
settings can be changed while a task is running.

Parameters for a task can be specified at any time; if a task needs information
that has not been specified yet, the user is prompted.

Context-sensitive help about tasks is provided at a single keystroke.

Tasks keep the user informed in two ways: they write in a log file
in which the user can page and search and they can provide a one-line status
message, which can be frequently updated.

Two versions of \H\ exist:
\begin{itemize}
\item \tH, which runs on standard character display terminals.
This version is used for interactive work.
\item \nH, a non-interactive version intended for batch work.
\end{itemize}
Normally \H\ is not started directly by the user, but rather by a shell script
which prepares a number of settings before starting \H. Refer to the
\G\ users guide.
\section*{Conventions in this document}

Text that appears literally on the terminal or workstation  screen such as
names of programs and commands is presented in {\tt typewriter} font.
Text which designates variable information is presented in $italic$ font.

\begin{quote}
\small Text, presented like this (smaller type and indented margins), contains
detailed information and may be skipped at first reading.
\end{quote}
\chapter{Tasks}

\section{Introduction}
Tasks in \G\ are the application programs which do useful work for the user.
They consist of one or more processes (usually one) which communicate with
the user through a formalized collection of interface routines.
Because of the use of these interface routines, tasks can be run from any
version of \H.

\begin{quote}
\small Though tasks can be run from all versions of \H, not all tasks
can be run {\em successfully} from all versions because a version of \H\ might
not implement all interface functions.
For instance the {\tt EDITFILE} function is not
implemented by \nH.

{\tt COLA}-scripts can be started as tasks. For information
about the use of {\tt COLA} refer to the document {\tt cola.dc1}.
\end{quote}

\section{User input}
\subsection*{Introduction}
Parameters for a task are passed to it via keywords.
A keyword is a character string followed by an equal sign, e.g.\ {\tt INSET=}.
Keyword names are case-insensitive, i.e.\ uppercase and lowercase letters
have the same meaning.
Parameters for a task can be specified in the command that starts the task and
thereafter at any moment while the task is active.
A task requiring input from the user causes \H\ to find out whether it has
already been specified and if not, prompt the user.
Before a parameter value is passed to a task, \H\ first checks whether it
meets the task's request and if not, \H\ rejects that value and prompts again.

As the execution of the task
proceeds, \H\ builds up a table of the task's keywords and associated
values.
At task termination this table is saved in order to allow the user to
run the task again with (partly) the same parameters.

Three classes of keywords are recognized:
\begin{enumerate}
\item
{\em Forced\/}, which are keywords for which no task default is allowed.
The user must respond to the prompt with a correct value.
\item
{\em Defaulted\/},
which are keywords which have a default value defined by the
task. If the keyword has not already been specified then the user
will be prompted.
If on this occasion the user types only a {\tt RETURN}, this
signifies that the task default is acceptable and should be used. Any
other input will override the default.
\item
{\em Hidden\/},
which are keywords for which the user is not prompted and
which have a default value defined by the task.
They can be specified by unprompted input by the user.
The documentation of a task must be read to find out if it uses
hidden keywords.
\begin{quote}
\small The user can set a parameter in the task context to change requests
for hidden keywords into requests for defaulted keywords. (``Unhide'' hidden
keywords.)

Hidden keywords are sometimes used to control looping in a task.
Each cycle round the loop, the task requests the value of a hidden keyword.
If it has not been specified, the task takes the default and looping continues.
If the user defines the keyword then that value is sent to the task which
could cause the task to terminate the loop.
\end{quote}
\end{enumerate}

\subsection*{Parameter syntax}
Basically \H\ is capable of providing tasks with the following types of
parameters:
integer numbers, real numbers, double precision reals, logicals and
character strings.
Tasks may however request input in the form of character strings and perform
their own decoding to obtain numerical information.
One parameter value can contain one or more elements.

\subsubsection*{Character input}
Depending on what the task specifies, character input can be given in two ways:
as {\em text\/} or as an {\em array\/}. When text is requested, all characters
belonging to the keyword, including blanks and comma's, are treated as a single
element.
In the case of an array, the elements must be separated by blanks or comma's.
If the input should contain special characters like (leading) blanks or
equals signs, it can be quoted using back-quotes ({\tt `}).

Normally character input is case-sensitive,
i.e.\ uppercase and lowercase letters have different meanings,
but individual tasks can deviate
from this or convert to either uppercase or lowercase.

\subsubsection*{Number input}
Floating point or integer numbers can be typed as numbers and/or
expressions (see also section ``Number input -- operators,
constants and functions'').
They must be separated by blanks or comma's.

Lists of numbers can be specified using the
`$start${\tt :}$end${\tt :}$increment$'
notation, where the `{\tt :}$increment$' part is optional and defaults to one.
A list of {\it n} identical values can be specified as
`{\it value\/}{\tt ::}{\it n}.'
A list can also be used as an operand in an expression.
In this case the list must be enclosed by square brackets {\tt [ ]} or
parentheses {\tt ( )}.
The expression is then evaluated for each element of the list.

The operator {\tt ?} can be used to select one or more items from a list.
See example below.

Examples:

{\tt
\begin{tabular}{lll}
1 2 3/3  sin(pi)&{\rm yields}&       1.0 2.0 1.0 0.0\\
log(10)::4&{\rm yields}&       1.0 1.0 1.0 1.0\\
log(10):log(100):2/4&  {\rm yields}&       1.0 1.5 2.0\\
10**[0 1 5]     &      {\rm yields}&      1 10 100000\\
{}[1:3]+[90:70:-10]  &     {\rm yields}&       91 82 73\\
{}[20:30]?[3 4 5] & {\rm yields}& 22 23 24\\
\end{tabular}
}

Suppose an inclination at keyword {\tt INCL=} must be given in
degrees, but you want to give the input in axis ratio.

{\tt INCL= deg(acos(0.3 0.5))}

The argument for {\tt acos} is a list consisting of the elements 0.3 and 0.5.
The expression evaluates these two values for {\tt acos} and converts
the resulting values to degrees. In this way the values 0.3 and 0.5
are converted to the angles $72.5424^\circ$ and $60.0^\circ$.

\subsubsection*{Logical input}
Logicals are decoded in the following way: {\tt YES}, {\tt JA} and {\tt TRUE}
result
in a logical which is true; {\tt NO}, {\tt NEE} and {\tt FALSE}
give a logical which
is false. It is sufficient to give the first letter of the possible
affirmative and negative replies. Any other answer will result in
a syntax error.

Examples:
\begin{flushleft}
{\tt
PLOTGRIDS= N\\
OK= YES\\
}
\end{flushleft}
%For logicals the inputs {\tt YES}, {\tt JA} or {\tt TRUE}
%are affirmative (the first letter is sufficient). Other inputs result in
%{\tt NO}.

\begin{quote}
\small
{\bf Input Lists and Recall Files}

If a task requests a keyword repeatedly, the user can pre-specify inputs in
two ways. The first method is separating the inputs with semicolons,
e.g.

{\tt POS= 10 20; 15 25; 30 30}

Note that a trailing semicolon
designates an empty input for which a default is to be taken.
The second method consists of specifying a file
(``{\em recall file\/}'') as input to the keyword.
This must be a text file  with a name like ``{\it name\/}{\tt .rcl}''.
Every line in this file is a separate input. Lines must not be longer than 500
characters.
Either the whole file or a part of it can be specified:

\begin{tabular}{ll}
{\tt <}$name$&use the whole file;\\
{\tt <}$name$ $n${\tt:}$m$&use line numbers $n$ to $m$;\\
{\tt <}$name$ {\tt:}$n$&use line numbers 1 to $n$;\\
{\tt <}$name$ $n${\tt:}&use line numbers $n$ to the end of the file;\\
{\tt <}$name$ $n$ &use only line number $n$.\\
\end{tabular}

Here the $n${\tt:}$m$ range specifier should not be confused with a list of
numbers as described above.

It is possible to specify a recall file in a semicolon-separated list, but
within a recall file semicolons cannot be used. Recall files cannot be used
recursively, i.e.\ a recall file can not contain a reference to an other
recall file.

When the task has ``consumed'' all pre-specified input on a keyword, the next
request for that keyword will cause the user to be prompted in the normal way.
Inputs given on repeatedly requested keywords are stored in a
semicolon-separated list, so if a task is re-run with the previous set of
parameters, {\em all\/} inputs are available again.

If unprompted input is given on a keyword of which all pre-specified input has
not been consumed yet, the new input will supersede the {\em complete\/}
previous set of inputs.

\newpage
{\bf Default files}

Keywords can be prespecified in a text file with a name like
``{\it name\/}{\tt .def}''. This {\em default file\/} can contain any
number of {\it keyword\/}{\tt =}{\it value}\/ pairs.
Any user supplied parameters will supersede those obtained from a default file.

The name of the {\em task} determines which default file will be read, not the
name of the executable. This has consequences (and possibilities) for tasks
which are run under an alias name.

It is also possible to specify defaults to be used by every task. For this
purpose the file {\tt tasks.def} us used. Such global defaults have
a limited use. The keyword {\tt GGIOPT=}, used by tasks with a graphical user
interface, is an example.

Default files can reside in both the current working directory and the
directory {\tt .gipsy} under the user's
login directory. Keywords from default files in the working directory supersede
identical keywords from {\tt .gipsy}. And keywords from task-specific
default files supersede identical global keywords.

\end{quote}

\clearpage
\subsubsection*{Number input -- operators, constants and functions}
In numerical expressions the following operators are available:

{\small
\begin{tabular}{ll}
{\tt +}&addition \\
{\tt -}&subtraction \\
{\tt *}&multiplication \\
{\tt /}&division \\
{\tt **}&exponentiation \\
\end{tabular}
}

The following predefined constants are available:

{\small
\begin{tabular}{ll}
{\tt PI}&$\pi$ (3.14159....) \\
{\tt C}&speed of light (SI) \\
{\tt H}&Planck (SI) \\
{\tt K}&Boltzmann (SI) \\
{\tt G}&gravitation (SI) \\
{\tt S}&Stefan-Boltzmann (SI) \\
{\tt M}&mass of Sun (SI) \\
{\tt P}&parsec (SI) \\
{\tt BLANK}&universal undefined value \\
\end{tabular}
}

The following mathematical functions can be used:

{\small
\begin{tabular}{ll}
{\tt abs({\it x})}&absolute value of {\it x}\\
{\tt sqrt({\it x})}&square root of {\it x}  \\
{\tt sin({\it x})}&sine of {\it x}\\
{\tt asin({\it x})}&inverse sine of {\it x}\\
{\tt cos({\it x})}&cosine of {\it x}\\
{\tt acos({\it x})}&inverse cosine of {\it x}\\
{\tt tan({\it x})}&tangent of {\it x}\\
{\tt atan({\it x})}&inverse tan of {\it x}\\
{\tt atan2({\it x},{\it y})}&inverse tan (mod 2$\pi$)\\
 &{\it x} = sin, {\it y} = cos\\
{\tt exp({\it x})}&exponential of {\it x} ($e^x$)\\
{\tt ln({\it x})}&natural log of {\it x}\\
{\tt log({\it x})}&log (base 10) of {\it x}\\
{\tt sinh({\it x})}&hyperbolic sine of {\it x}\\
{\tt cosh({\it x})}&hyperbolic cosine of {\it x}\\
{\tt tanh({\it x})}&hyperbolic tangent of {\it x}\\
{\tt rad({\it x})}&convert {\it x} to radians\\
{\tt deg({\it x})}&convert {\it x} to degrees\\
{\tt erf({\it x})}&error function of {\it x}\\
{\tt erfc({\it x})}&1-error function\\
\end{tabular}
\begin{tabular}{ll}
{\tt max({\it x},{\it y})}&maximum of {\it x} and {\it y}\\
{\tt min({\it x},{\it y})}&minimum of {\it x} and {\it y}\\
{\tt sinc({\it x})}&sin({\it x})/{\it x} (sinc function)\\
{\tt sign({\it x})}&sign of {\it x} ($-1$, 0, 1)\\
{\tt mod({\it x},{\it y})}&remainder of {\it x}/{\it y}\\
{\tt int({\it x})}&truncates to integer\\
{\tt nint({\it x})}&nearest integer\\
{\tt ranu({\it x},{\it y})}&generates uniform noise between\\
&{\it x} and {\it y}\\
{\tt rang({\it x},{\it y})}&generates gaussian noise with\\
&mean {\it x} and dispersion {\it y}\\
{\tt ranp({\it x})}&generates poisson noise with mean {\it x}\\
\\
{\tt ifeq({\it x},{\it y},{\it a},{\it b})}&
{\it a} if $x=y$, else {\it b} \\
{\tt ifne({\it x},{\it y},{\it a},{\it b})}&
{\it a} if $x \neq y$, else {\it b} \\
{\tt ifgt({\it x},{\it y},{\it a},{\it b})}&
{\it a} if $x > y$, else {\it b} \\
{\tt ifge({\it x},{\it y},{\it a},{\it b})}&
{\it a} if $x \geq y$, else {\it b} \\
{\tt iflt({\it x},{\it y},{\it a},{\it b})}&
{\it a} if $x<y$, else {\it b} \\
{\tt ifle({\it x},{\it y},{\it a},{\it b})}&
{\it a} if $x\leq y$, else {\it b} \\
\end{tabular}
}

The following functions can be used to obtain data from \G -sets and files:

{\small
\begin{tabular}{ll}
{\tt descr({\it set}, {\it name})}&descriptor item {\it name}\/ from
(sub)set(s) {\it set}.\\
{\tt table({\it set}, {\it tab}, {\it col}, {\it rows})}&
cell(s) from column {\it col}\/ of table {\it tab} in (sub)set {\it set}.\\
{\tt image({\it set}, {\it box})}&  pixel(s) from (sub)set {\it set}.\\
{\tt file({\it name}, {\it cols}, {\it rows})}&number(s) from a column in 
a text file.\\
\end{tabular}
}

The {\it set}- and {\it box}\/ arguments have the same syntax as described
in {\tt input.doc}.
The argument {\it rows} has the same syntax as is used for recall files
($n${\tt:}$m$, {\tt:}$n$, $n${\tt:} and  $n$).
Lines in the file to be read by the {\tt file}-function must not be longer than
500 characters.
Comments can be included in the file by prefixing it by one the characters
{\tt !} or {\tt \#}.
All items in the file should be numbers.
If an item cannot be recognized, the value {\tt BLANK} will result.
In this way {\tt BLANK}s can also be specified explicitly, e.g. by using
the word ``BLANK''. Empty columns at the end of a line will also yield
{\tt BLANK}s. But a completely empty or blank line is treated as a comment.

The following functions can be used on lists:

{\small
\begin{tabular}{ll}
{\tt count($x$)}&number of elements in $x$\\
{\tt mean($x$)}&the average of the elements in $x$\\
{\tt sum($x$)}&the sum of the elements in $x$\\
\end{tabular}
}

\H\ also supports named variables.
To these variables numbers and lists of numbers can be assigned. 
Variable names can be up to 20 characters long; they are not case-sensitive.
The maximum number of variables is 40.
A variable can be deleted by assigning the value {\tt BLANK} to it.
Assignment to a variable is accomplished by issuing a special \H\ command:

{\it variable}{\tt =}{ \it value}

or, in \tH, doing the assignment in the \tH\ calculator window (see page
\pageref{calculator}).

\section{Task context}
Tasks all run in a {\em context\/}. This context consists of a number of
user-settable parameters:
\begin{itemize}
\item
Error level. If a task generates an error at or above the current error level,
\H\ will abort the task. Default: 4.
\item
Message level. If a task generates an error at or above the current message
level,
\H\ will display the associated error message. Default: 1.
\item
Output mode. This parameter determines whether test messages from tasks will be
displayed and whether messages unnecessary for experienced users will be
suppressed. Default: ``{\tt NORMAL}''.
\item
Device status. Two parameters indicating whether messages sent to either the
screen or the log file will actually be written.
Default: screen ``{\tt ON}'' and log file ``{\tt ON}''.
\item
Hide status. This parameter determines whether ``hidden'' user input requests
from the task will cause the user to be prompted anyway. Default: ``{\tt ON}'',
i.e.\ the user will not be prompted for hidden keywords.
\item
Working directory. Default: the directory from which \H\ was started.
The directory of running tasks cannot be changed.
\item
Finally there are four parameters which determine what kind of status messages
will be logged.
\end{itemize}
Tasks normally inherit their context from a template context. If a task is
started by an other task, it inherits the context of that task.
Both the template context and the context of any running task can be changed by
the user. The way in which this can be done depends on the specific
implementation of \H.
When \H\ is started, the template context contains the sensible defaults
listed above, possibly modified by parameters from the \H\ defaults file
(see below).

\section{\H\ defaults file}
The template task context and some other parameters can be defined in a
defaults file which will be read when \H\ is started.
This file must have the name {\tt hermes.def} and reside in the directory
from which \H\ is started or in the directory {\tt .gipsy} under the user's
login directory. (Parameters
from the file in the start-up directory have precedence.)
Its format is a text file which may contain any number of the keyword-value
pairs from the table below.

\begin{tabular}{ll}
{\tt DIRECTORY=}&working directory for tasks.\\
{\tt TASKPATH=}&search path for tasks.\\
{\tt NTASK=}&number of task entries; must be $>0$.\\
{\tt HIDE=}&hide status for keywords; must be {\tt ON} or {\tt OFF}.\\
{\tt ERRORLEVEL=}&error level; must be in the range 1 to 4.\\
{\tt MESSAGELEVEL=}&message level; must be in the range 1 to 4.\\
{\tt OUTPUTMODE=}&
output mode; must be {\tt NORMAL}, {\tt EXPERT} or {\tt TEST}.\\
{\tt SCREEN=}&screen output status; must be {\tt ON} or {\tt OFF}.\\
{\tt LOGFILE=}&log file output status; must be {\tt ON} or {\tt OFF}.\\
{\tt PRINTER=}&printer number for screen hardcopies.\\
{\tt MATCH=}&minimal matching; must be {\tt ON} or {\tt OFF}.\\
{\tt TEKDEVICE=}&Tektronix terminal type (\tH\ only).\\
{\tt BEEP=}&terminal bell; must be {\tt ON} or {\tt OFF} (\tH\ only).\\
{\tt KEYHELP=}&
add keyword help to standard help; must be {\tt ON} or {\tt OFF}.\\
{\tt WEB\_BROWSER=}&full-path location of Hypertext browser and optional
command line arguments.\\
{\tt STARTBROWSER=}&start Hypertext browser automatically.\\
\end{tabular}

\vspace{1cm}
Example of a typical {\tt hermes.def} file:

\begin{quote}
\begin{verbatim}
taskpath=. $HOME/owntasks/$gip_arch
directory=/dj3/users/cia/gipsy
outputmode=expert
match=on keyhelp=off
printer=3
\end{verbatim}
\end{quote}

\chapter{\tH}
\section{Introduction}
\tH\ is the version of \H\ which communicates with the user through standard
character display terminals, including X11 terminal windows.
It organises the terminal screen in a special window-like way.
The most important permanent components (windows) of the screen lay-out are:
\begin{itemize}
\item
User Command Area (\UCA ). This occupies the lower two lines of the terminal
screen and is the focus of user-to-task communication.
\item
Task Status Area (\TSA ). This normally consists of four lines above the {\UCA}
which each can contain information reflecting the current status of a task.
\item
Common Output Area (\COA ), the remaining top part of the screen which
shows a part of the log file.
Here user commands are logged and tasks can write text.
\end{itemize}
Other permanent components are a clock display and numbers which indicate
which part of the log file is mapped to the \COA .

Components of the screen lay-out which are not permanent are called transient
windows.
These overlay the permanent screen components or other transient windows.
For example the context-sensitive help display overlays (part of) the \COA .
Other examples of transient windows are
a menu window allowing the user to change specific task settings, prompter
windows in which for instance a search string can be specified, a Unix shell
window, etc.

Many functions of \tH\ can be invoked by typing a control key (designated
as {\tt CTRL-}{\it c}, where {\it c} is a letter), or by typing an escape
sequence (the {\tt ESC} character followed by another character).
These are summarized in the tables on pages \pageref{escseqs} and
\pageref{ctrlkeys}.
Important control keys are: {\tt CTRL-C} to abort
a task or to break out of a transient window, {\tt CTRL-O}\label{ctrlo}
to switch the
``keyboard focus'' between terminal windows waiting for keyboard input,
{\tt CTRL-L}\label{ctrll} to repair a damaged screen lay-out
and {\tt CTRL-Q}\label{ctrlq} to quit from \H.
\H\ can also be terminated by typing {\tt CTRL-X CTRL-C}\label{ctrlxctrlc} or
{\tt ESC Z}\label{escz} or by typing {\tt QUIT}, {\tt STOP}, {\tt END} or
{\tt EXIT} in the \UCA.

It is not possible to run more than one \G\ session in the same directory.
If \tH\ is started in a directory where \G\ already runs, it exits
immediately with an error message.
(But \nH\ will wait until the other session is finished.)

\section{User Command Area}
The User Command Area (\UCA ) consists of the bottom two lines of the terminal
screen where most interaction with the tasks takes place.
The user can type task-related commands in the \UCA\ and \H\ can put there
task-related prompts.
An important aspect of the prompts is that they
are treated in exactly the same way as information typed by the user, i.e.\ the
user can erase or modify a prompt supplied by \H\ as if he typed it himself.
Prompts in the \UCA\ are always in the form of a legal complete or partial
command.

The information in the \UCA\ is used as the context for a number of functions of
\H.
E.g.\ {\tt TAB} (or {\tt CTRL-I}) causes the user document of any taskname
present in the
\UCA\ to be displayed and {\tt CTRL-C} aborts the task if it is active.

The two lines of the \UCA\ are one logical long line which can be edited using
the following control keys of which most are {\sc emacs}-compatible.

\label{ctrla}\label{ctrlb}\label{ctrld}\label{ctrle}\label{ctrlf}
\label{ctrlk}\label{ctrlt}\label{ctrlu}
\begin{tabular}{ll}
{\tt CTRL-A}&move to start of line\\
{\tt CTRL-B}&move one position back\\
{\tt CTRL-D}&forward delete one character\\
{\tt CTRL-E}&move to end of line\\
{\tt CTRL-F}&move one position forward\\
{\tt CTRL-H}&backward delete one character\\
{\tt CTRL-K}&forward delete rest of line\\
{\tt CTRL-T}&replace complete line with next active task name (see below)\\
{\tt CTRL-U}&delete complete line\\
{\tt DEL}&backward delete one character\\
{\tt space}&replace complete line with next prompt for user input (see below)\\
\end{tabular}

\begin{quote}
\small
\parskip=0mm
Characters {\tt CTRL-T} and {\tt space} are somewhat special.
{\tt CTRL-T} erases the \UCA\ and then prompts with the name of an active task.
It does this in a circular fashion so if more than one task is active,
another taskname is displayed when {\tt CTRL-T} is pressed again.

If the \UCA\ is ``free'' (a subtle concept explained below), pressing the space
bar does a similar thing for tasks waiting for input. This makes it easier for
the user to do something else before he supplies the requested input, e.g.
run another task which supplies information that helps to give the input.

The \UCA\ is ``free'' when it is empty, or when it only contains an unmodified
prompt supplied by \H.
\end{quote}

\H\ can provide `minimal matching' for task names and user input keywords
typed in the \UCA.
This facility can be switched on and off by typing {\tt ESC M}.\label{escm}
It can also be switched on using the statement {\tt MATCH=ON} in the
\H\ defaults file.
If it is on, task names will be completed when a blank
or {\tt RETURN} is typed on a command line that only
contains an abbreviated task name.
Keywords will be completed when `{\tt =}' is
typed. If the keyword is ambiguous, \H\ will beep and refuse to accept the
keyword; if it is not known, \H\ will also beep but will accept the
keyword.\newline
\H\ obtains the taskname and
keyword information from the {\tt .dc1} documents in {\tt \$gip\_tsk}.
For this reason minimal matching does not work for tasks which are not
installed.

Previously entered commands can be brought back to the \UCA\ by using the
\label{uparrow}\label{downarrow} ANSI up-arrow and down-arrow keys.
If the user has typed some text in the
\UCA, only commands which match this text will be brought back.
The commands are saved in the text file {\tt HERMES.cmd}, so that they
are preserved between sessions.

\section{Task Status Area}
The Task Status Area (\TSA ) is the middle section of the terminal screen.
It normally consists of four lines which each can contain status information
of an active task.
\begin{quote}
\small
\parskip=0mm
The number of these status lines can be specified using the keyword
{\tt NTASK=} in {\tt hermes.def}.
\end{quote}
The format of a task status entry is one of the following:
\begin{itemize}
\item When a task has been activated, but is not running yet:\\
{\it taskname \hspace{5mm} \tt WAITING TO BE RUN}
\item When a task is running:\\
{\it taskname \hspace{5mm} \tt RUNNING}\\
or\\
{\it taskname \hspace{5mm}  status message supplied by  task}
\item When a task is prompting for user input:\\
{\it taskname \hspace{5mm}  prompt message supplied by  task}
\item When a task is waiting for a file to be edited by the user:\\
{\it taskname \hspace{5mm}  message supplied by  task}
\item When a task is suspended:\\
{\it taskname \hspace{5mm} \tt PAUSING }\\
or\\
{\it taskname \hspace{5mm} \tt PAUSING \it message supplied by  task}
\item When a task is waiting for an other task to finish:\\
{\it taskname \hspace{5mm} \tt WAITING FOR \it other taskname}
\item When a task has finished processing normally:\\
{\it taskname \hspace{5mm} \tt +++ FINISHED +++}
\item When a task has been aborted by the user:\\
{\it taskname \hspace{5mm} \tt USER ABORT}
\item When a task has encountered an error fatal to execution:\\
{\it taskname \hspace{5mm} error message \tt -FATAL}
\item When a task has crashed (i.e.\ exited without notification to \H):\\
{\it taskname \hspace{5mm} \tt CRASHED}
\item When a task cannot be started (incorrect executable,
e.g.\ compiled for a different architecture):\\
{\it taskname \hspace{5mm} \tt COULD NOT BE STARTED}
\end{itemize}

\section{Common Output Area and Log File}
The Common Output Area (\COA ) on the terminal screen is the remaining top area
of the terminal screen not used for the \UCA\ or \TSA .
It is a viewport on the log file.
\H\ treats the log file as a ``book'' with numbered pages.
Normally \H\ shows the current page of the book, that is the page on
which output is currently being written.

The \COA\ has two modes of operation: page mode and non-page mode.
In non-page mode the page on the screen is always the current page.
If necessary, \H\ will flip the page. The current page number is displayed at
the extreme right of the Task Status Area.
In page mode the information on the screen will stay there until the user
instructs \H\ to change the page.
Though the user does not see it, output can still be written to the log file.
In page mode the number of the displayed
page is shown to the left of the current page number.
Page mode can be switched on and off by pressing {\tt CTRL-P}.\label{ctrlp}

When in non-page mode, \H\ can also scroll the \COA\ instead of flipping
pages. This scroll mode can be switched on and off by typing \label{ctrlxs}
{\tt CTRL-X S}.
In scroll mode output is much slower than in `flip mode'. Therefore this
mode is not recommended for use with slow terminals.

The log file can be traversed in backwards direction by pressing
the {\tt CTRL-Z}\label{ctrlz} key (or on ANSI keyboards the
{\tt PageUp} key).
Paging forward is done with the {\tt CTRL-V}\label{ctrlv}\label{ctrlj} key
(or ANSI {\tt PageDown}).

Text search in the log file is done by pressing {\tt CTRL-R}\label{ctrlr}
(reverse) or
{\tt CTRL-S}\label{ctrls}
(forward). \H\ then prompts the user for a search string.
Searching in \H\ is case-insensitive.

A specific page number can be brought to the screen by typing {\tt ESC G},
\label{escg} whereafter \H\ prompts for a page number.
Signed numbers are treated as relative page numbers.

Sections of the log file can be printed by typing the {\tt ESC H}\label{esch}
sequence.
\H\ then prompts for a range of pages to be printed.
The range can be specified as a single number, in which case only that page
will be printed, or using the format `{\it start\/}{\tt :}{\it end\/}'.
Either {\it start\/} or {\it end\/} may be omitted, in which case the number
of the currently displayed page will be used instead.
The default range is the page currently displayed and the preceding two pages.
The printer can be selected by typing {\tt ESC P}, \label{escp} which causes
a menu to be presented from wich a printer can be chosen.
Instead of a printer, output can also be written to a file. To achieve this,
choose option {\tt F} from the printer selection menu. This redirection
can be cancelled by specifying an empty string.

\begin{quote}
\small
The statement that the \COA\ is a viewport on the log file is not quite true.
In fact the log file and the \COA\ are different output streams to which tasks
can send information selectively.
Most tasks however send the same information to both the log file and the \COA .
\end{quote}

\subsection*{COA command summary}

\begin{tabular}{ll}
{\tt CTRL-P}&enter or leave page mode\\
{\tt CTRL-Z}&display previous page and enter page mode\\
{\tt CTRL-V}&display next page\\
{\tt CTRL-R}&search forward for text string\\
{\tt CTRL-S}&search backwards for text string\\
{\tt CTRL-X S}&toggle scroll mode\\
{\tt ESC G}&go to specified page number\\
{\tt ESC H}&make hardcopy of specified page numbers\\
{\tt ESC P}&select a printer for the hardcopy print command\\
{\tt ESC <}&go to first page\\
{\tt ESC >}&go to last page\\
\end{tabular}

\section{Running Tasks}
\subsection*{Starting tasks and supplying parameters}
Tasks can be started with or without specifying parameters.
To start a task without parameters, the name of the task must be typed in the
\UCA , followed by a {\tt RETURN}:

{\it taskname}

To specify parameters, any number of keyword-value pairs separated
by blanks can be added to the taskname:

{\it taskname
keyword\/${}_1${\tt =}value\/${}_1$
keyword\/${}_2${\tt =}value\/${}_2$ $\ldots$}

This format is also used to supply parameters to an already running task.

If there is something wrong with the command, an error message is displayed
in the lower right corner of the screen.

\begin{quote}
\small
Tasknames may be preceded by an explicit directory. In this case, the task must
be present in the specified directory; the task path is not searched.
The directory may contain one or more environment variables
(indicated by a leading {\tt \$}-character).

Tasks can be run again with (partly) the same parameters by prefixing the
task start command with an exclamation mark ({\tt !}).
Any explicitly specified parameters will supersede the corresponding old
parameters. In this case the default file is not used.
For historical reasons the previous parameters of a task are called the
task's {\em macro\/}.

Tasks can be run under a different name (``alias''). The command to do this is:

{\it aliasname\tt (\it filename\tt ) $\ldots$}

{\it Aliasname} is the taskname under which the task will run and
{\it filename\/} is the name of the task's executable file, specified in the
correct case (lowercase for standard \G\ tasks).
\end{quote}

\subsection*{Inspecting and saving input parameters}
The current set of keyword-value pairs of a task can be inspected by
typing {\tt ESC K}\label{esck} while the name of the task
is present in the \UCA.
This causes (part of) the \COA\ to be overlaid with a display containing the
current user input parameters of the task.
Whenever there is a change in one of the parameters, the display will
be adjusted.
If the amount of space on the screen is not sufficient to show all parameters,
the user can page through the display just like paging through the log file.
{\tt CTRL-V}\label{ctrlvv} pages forward; {\tt CTRL-Z}\label{ctrlzz} pages
backwards.
To inspect its keywords, the task need not be active.
Typing {\tt ESC K} again will remove the overlay screen.

The displayed set of parameters can be saved in a file which can later be used
as a default file. This can be achieved by typing {\tt ESC W},
\label{escw} which causes
the user to be prompted for a filename (default: {\it taskname\/}{\tt .def}).

The contents of a single keyword from the displayed set of parameters
can be saved in a file which can later be
used as a recall file. Type {\tt ESC R}.
\label{escr}The user is first prompted for the
keyword, then for the filename.

The parameters of a task can be edited using a standard editor.
The editor used is the one defined by the {\tt EDITOR} environment variable
and it is invoked by typing {\tt ESC E}.\label{esce}
If there is currently a set of parameters displayed on the screen (as the
result of typing {\tt ESC K}), these parameters will be edited;
otherwise the name of the task is derived from the \UCA\ contents.
`Behind the scenes' \H\ will remain active, but user communication with \H\
is not possible because all typing goes into the editor.
Using this editor is the only way in \tH\ to change input parameters for
a task that is not active.\newline
If \H\ cannot derive a name from either the parameter display or the \UCA,
it will prompt for a file name as if {\tt CTRL-X E} was typed.

\subsection*{Aborting, suspending and resuming}
Tasks can be aborted by typing {\tt CTRL-C}\label{ctrlc}.
They can be suspended and resumed by typing {\tt CTRL-W}\label{ctrlw} or
{\tt CTRL-G}\label{ctrlg}.
If more than one task is active, the target task's name must be present in
the \UCA .

\subsection*{Changing the task context}
The context of a task can be changed using a pop-up menu that can be activated
with the sequence {\tt ESC S}.\label{escs}
The following context parameters can be changed with this menu:
\begin{verse}
- error level: 1--4\\
- message level: 1--4\\
- output mode: normal, expert or test\\
- screen output: on or off\\
- log file output: on or off\\
- hide ``hidden'' keywords: on or off\\
\end{verse}
To change the context of a running task, its name must be present in the \UCA.
If the \UCA\ is empty, the template context will be changed.

\subsection*{Directories and paths}
The working directory can be changed by typing {\tt ESC D}.\label{escd}
\H\ then prompts the user for a new directory.
This directory will then be used for tasks started after the change.
Running tasks and \H\ itself are not affected.

The task path can be changed with {\tt ESC T}.\label{esct}
Upon the prompt that follows one or more directories
separated by blanks can be specified.
A directory may contain (or consist of) one or more environment variables
(indicated by a leading {\tt \$}-character).
Whenever a task start command without an explicit path is given,
this list is scanned to find the task.
If it is not in one of the specified directories, \H\ tries to find the task in
{\tt \$gip\_exe}.
The default path is `{\tt .}' (the current working directory).

\subsection*{Getting help}
To obtain help information about a task or about \H, {\tt TAB} (or {\tt CTRL-I})
\label{ctrli} can be pressed.
This causes (part of) the \COA\ to be overlaid with a display containing the
user document of the task of which the name is present in the \UCA .
If there is also a keyword present (e.g.\ as the result of a prompt),
\H\ positions the document at the first occurrence of the keyword.
In this case \H\ will also provide a keyword help window which partially
overlays the task documentation. For more information see the section
``Keyword help'' below. This feature can be switched off by putting the
statement {\tt KEYHELP=OFF} in the \H\ defaults file ({\tt hermes.def}).

The help display will `follow' the task execution; if other prompts appear,
the display will be repositioned.
If the amount of space on the screen is not sufficient to show the whole
document,
the user can page through the display just like paging through the log file.
{\tt CTRL-V}\label{ctrlvvv}  pages forward; {\tt CTRL-Z}\label{ctrlzzz}
pages backwards.
If the \UCA\ is empty, pressing {\tt TAB} will display
a summary of \H' commands.

It is possible to search for a text string in the document. Pressing
{\tt CTRL-S}\label{ctrlss} or {\tt CTRL-R}\label{ctrlrr} will cause the
user to be prompted for a search string.

Pressing {\tt TAB} again will remove the overlay screen(s).

\subsubsection*{Hypertext help}
Typing {\tt ESC X}\label{escx} will cause Hermes to start a WWW browser.
The \G\ start-up script attempts to determine which browser can be used.
This choice can be overridden by defining the environment variable
{\tt WEB\_BROWSER} before starting \G\ or by defining the variable
{\tt WEB\_BROWSER} in the \H\ defaults file ({\tt hermes.def}).
Any command line options to the browser can also be specified in the
defaults file (but {\em not\/} in the environment variable).
When the defaults file variable {\tt STARTBROWSER} is set to ``{\tt ON}'',
the browser will be started automatically. 

Initially the home page of the new-style \G\ documentation in hypertext
will appear. When the browser is either NCSA Mosaic, Netscape or Firefox,
pressing {\tt TAB} will display the task document in the browser instead 
of overlaying the \COA.

When {\tt ESC X} is typed again, the browser will be terminated and \H\ will
revert to normal operation.

\subsection*{Keyword help}
\H\ can provide specific help about user input keyword prompts.
This can be switched on and off independently of task-related help
by typing {\tt ESC TAB}.
When it is active, the available options are displayed in the bottom line of
the window. The possible options are:

\begin{tabular}{ll}
{\tt PageUp}&page back in window\\
{\tt PageDown}&page forward in window\\
{\tt ESC $i$}&go to reference $i$ ($i=1\ldots 9$)\\
{\tt ESC -}&go back one reference\\
{\tt ESC 0}&go back to ``root text''\\
\end{tabular}

The window can be temporarily removed by removing the associated prompt by
typing {\tt CTRL-U}. When the prompt is brought back by typing a space, the
window will re-appear.

\subsection*{Task control summary}
\begin{tabular}{ll}
{\it taskname}&start a task\\
{\it taskname
keyword\/{\tt =}value\/ $\ldots$}&start a task and/or supply parameters\\
{\tt !}{\it taskname}&start a task with the previous set of input parameters\\
{\tt !}{\it taskname
keyword\/{\tt =}value\/ $\ldots$}&like {\tt !}{\it taskname\/}, but given
parameters replace previous\\
{\it taskname}{\tt (}{\it filename\/}{\tt )} $\ldots$&
run executable {\it filename\/} as task {\it taskname}\\
\end{tabular}

\begin{tabular}{ll}
{\tt CTRL-C}&abort task\\
{\tt CTRL-G}&suspend or resume task (synonym of {\tt  CTRL-W})\\
{\tt CTRL-I}&(={\tt TAB}) activate or deactivate help display\\
{\tt CTRL-W}&suspend or resume task (synonym of {\tt  CTRL-G})\\
\\
{\tt ESC D}&change working directory\\
{\tt ESC K}&activate and deactivate user input parameter display\\
{\tt ESC E}&edit parameters or user-specified file\\
{\tt ESC W}&save displayed parameters in (default) file\\
{\tt ESC R}&save contents of keyword in (recall) file\\
{\tt ESC S}&activate menu to modify task context\\
{\tt ESC T}&change task path\\
{\tt ESC TAB}&switch user input keyword help display on or off\\
\end{tabular}
\section{Unix shell and text editor}
\subsection*{Integrated shell}
Typing {\tt ESC U}\label{escu} causes the \COA\ to be overlaid by a Unix
shell window or,
if the Unix shell is already present, causes the window to be popped down.
The shell used is {\tt /bin/csh}.
Because the shell is not connected to
a real terminal, functionality is somewhat limited: e.g.\ a screen editor
would not work.
Also some commands behave differently, e.g.\ {\tt ls} produces a one-column
list. (A multi-column list can be obtained with {\tt ls -C}.)

By default information displayed in the Unix shell window is not logged.
Logging to the \COA\ and log file can be enabled (and disabled) by typing
{\tt ESC L}.\label{escl}

This shell can be completely terminated by typing {\tt exit} in the shell
window.
\subsection*{Overlay shell}
Typing {\tt CTRL-X U}\label{ctrlxu} causes a Unix shell to be started which
overlays the whole terminal screen.
`Behind the scenes' \H\ will remain active, but because all typing goes into
the shell, user communication with \H\ is not possible.
The shell used is the one defined by the {\tt SHELL} environment variable or,
if this is not defined, {\tt /bin/csh}.

This shell can be terminated by typing {\tt CTRL-D} or {\tt exit}.
\subsection*{Text editor}
Typing {\tt CTRL-X E}\label{ctrlxe} causes the user to be prompted for the name
of a file to be edited by a text editor which will overlay the whole terminal
screen.
`Behind the scenes' \H\ will remain active, but because all typing goes into
the editor, user communication with \H\ is not possible.
The editor is the one defined
by the environment variable {\tt EDITOR};
if this is not defined, the \G\ start-up script will attempt to find a
one of the editors {\tt mem}, {\tt vi} or {\tt emacs} in the user's path.
If no editor can be found, an error message is displayed.

Tasks can also request that a text editor be started. In this case \H\ prompts
the user to allow the editor to be started. To start the editor,
type {\tt RETURN}; to cancel the edit, {\tt CTRL-C} can be typed and the
edit can be postponed by shifting the keyboard focus away from the prompt
window by typing {\tt CTRL-O}.

\section{Tektronix graphics}
If the terminal at which \H\ runs is capable of emulating a Tektronix
graphics terminal, {\tt CTRL-X T}\label{ctrlxt} can be typed to switch on
the graphics mode.
This mode can be used in connection with tasks for which the
plot device {\tt THERTEK} was selected.
If the terminal type has not been defined yet, the user will be prompted.
This information can also be specified in the \H\ defaults file
{\tt hermes.def}. In this case use: {\tt TEKDEVICE=}{\it terminaltype\/}.
Allowable terminal types are listed in the table below.

In Tektronix mode, \H\ continues to operate normally,
but user interaction is somewhat limited. To protect the
graphics image, characters typed are not echoed and the overlay Unix shell
and editors cannot be started.

Typing {\tt CTRL-X T} again will switch Tektronix mode off.
If possible, \H\ then switches back to normal terminal mode, but some
terminals do not support this; these terminals must be switched back
manually by the user.

\H\ `remembers' the contents of the graphics image, so if Tektronix mode is
switched on again, the image is restored.

\vspace{5mm}
\begin{center}
\begin{tabular}{|l|l|l|}
\hline
Type&&switch back\\
\hline
{\tt LADA}&KGB's LAT terminal emulator&automatic\\
{\tt NCSA}&NCSA Telnet&manual\\
{\tt XTERM}&X Window terminal emulator&automatic\\
\hline
\end{tabular}
\end{center}

\section{Calculator}
\label{calculator}
Typing {\tt ESC C} causes a calculator window to be popped up.
In this window, the user can type any numerical expression that would also
be acceptable as user input. The resulting value is displayed in the bottom
line of the window; for lists only the first and last value are shown.
It is also possible to assign a value to a variable.

The calculator window can be removed by typing {\tt ESC C} again.

\chapter{\nH}
\section{Introduction}
\nH\ (non-interactive Hermes) is the version of \H\ for a
non interactive (i.e.\ batch-)  environment.
It is derived from the interactive \tH.

Because the user is not available to be consulted,
default values for unspecified keywords will be taken if available.
If there is no default available, an error message will be written to the log
file and \nH\ will stop with the current command.
When a command is finished, \nH\ will start the next command if given.

If the timeout option is set, a timeout will occur after the user supplied
number of minutes and \nH\ will terminate the command it is
dealing with at that time.
The timeout works separately for each user command.

\nH\ maintains a log file to which a history of all operations is written.
This file can be inspected or printed afterwards.

\section{Using \nH}
The syntax for starting \nH\ is:
\begin{flushleft}
{\tt nhermes -l{\it logfile} -t{\it minutes} {\it command ... command}}
\end{flushleft}

The options may appear in any order and may be intermixed with the
commands. The {\tt -l} and {\tt -t} flags are optional.

The log file will always have the extension {\tt .LOG}
and it contains a history of all operations done and the information
generated by the tasks which have been run.
The default log file name is {\tt GIPSY.LOG}.

The purpose of the timeout option is to terminate the user command when it
uses more time than is thought to be reasonable, e.g.\ if it is executing an
endless loop due to a program or user error.

{\it command} can be a {\tt COLA} script or a name of a  task with their
parameters.
The name of the  task or the {\tt COLA} script with the
parameters must be suitably quoted to keep it together,
e.g.\ between double quotes. If this is not done,
\nH\ will handle them as if they were separate user commands.

Example:

{\tt nhermes -lMylog "FLUX INSET=cg1517 v"}

This command causes \nH\ to start task {\tt FLUX} on all subsets of set
{\tt cg1517} and put the results in file {\tt Mylog.LOG}.

Status return:

If the last command executed by \nH\ was successful, \nH\ exits with status
code 0; when the last command failed, the value 1 is returned. 
\appendix
\chapter{tHermes key definitions}
\subsection*{Escape sequences}
\label{escseqs}
\begin{tabular}{ll}
{\tt ESC B}&switch beep associated with prompts on or off\\
{\tt ESC C}&pop up and down calculator window (\pageref{calculator})\\
{\tt ESC D}&change working directory (\pageref{escd})\\
{\tt ESC E}&edit a task's input parameters or a user-specified file
(\pageref{esce})\\
{\tt ESC G}&go to specified page number in \COA\ (\pageref{escg})\\
{\tt ESC H}&make hard copy of \COA\ (\pageref{esch})\\
{\tt ESC K}&pop up and down keyword display (\pageref{esck})\\
{\tt ESC L}&switch logging of integrated Unix shell on or off (\pageref{escl})\\
{\tt ESC M}&switch \UCA\ minimal matching on or off (\pageref{escm})\\
{\tt ESC P}&select printer for the {\tt ESC H} command (\pageref{escp})\\
{\tt ESC R}&(if keyword display active) write single task input parameter to
file (\pageref{escr})\\
{\tt ESC S}&change task or template context (\pageref{escs})\\
{\tt ESC T}&change task path (\pageref{esct})\\
{\tt ESC U}&pop up and down integrated Unix shell (\pageref{escu})\\
{\tt ESC W}&(if keyword display active) write all task's input parameters to
file (\pageref{escw})\\
{\tt ESC X}&activate or deactivate hypertext help (\pageref{escx})\\ 
{\tt ESC Z}&quit \tH\ (\pageref{escz})\\
{\tt ESC <}&go to first page of \COA \\
{\tt ESC >}&go to last page of \COA \\
{\tt ESC ESC}&circulate through task names in \UCA \\
{\tt ESC TAB}&switch user input keyword help display on or off\\
\end{tabular}

\begin{tabular}{ll}
{\bf ANSI keys}&\\
&\\
{$\uparrow$}&bring back previous commands to \UCA\ (backward direction)
(\pageref{uparrow})\\
{$\downarrow$}&bring back previous commands to \UCA\ (forward direction)
(\pageref{downarrow})\\
{$\leftarrow$}&move one position back in input line (={\tt CTRL-B})\\
{$\rightarrow$}&move one position forward in input line (={\tt CTRL-F})\\
{\tt PageUp}&page back in \COA, help display or keyword display
(={\tt CTRL-Z})\\
{\tt PageDown}&page forward in \COA, help display or keyword display
(={\tt CTRL-V})\\
\end{tabular}

\subsection*{Control characters}
\label{ctrlkeys}
\begin{tabular}{ll}
{\tt CTRL-A}&move to start of input line (\pageref{ctrla})\\
{\tt CTRL-B}&move one position back in input line (\pageref{ctrlb})\\
{\tt CTRL-C}&in \UCA : abort task; in transient window:
break out of window (\pageref{ctrlc})\\
{\tt CTRL-D}&forward delete character in input line (\pageref{ctrld})\\
{\tt CTRL-E}&move to end of input line (\pageref{ctrle})\\
{\tt CTRL-F}&move one position forward in input line (\pageref{ctrlf})\\
{\tt CTRL-G}&resume or suspend task (\pageref{ctrlg})\\
{\tt CTRL-H}&back delete one character\\
{\tt CTRL-I}&(={\tt TAB}) pop help screen up or down (\pageref{ctrli})\\
{\tt CTRL-J}&enter input line\\
{\tt CTRL-K}&forward delete rest of input line (\pageref{ctrlk})\\
{\tt CTRL-L}&rewrite terminal screen (\pageref{ctrll})\\
{\tt CTRL-M}&(={\tt RETURN}) enter input line\\
{\tt CTRL-N}&(undefined)\\
{\tt CTRL-O}&switch between windows accepting keyboard input (\pageref{ctrlo})\\
{\tt CTRL-P}&enter or leave \COA\ page mode (\pageref{ctrlp})\\
{\tt CTRL-Q}&quit \tH\ (\pageref{ctrlq})\\
{\tt CTRL-R}&reverse search in \COA\ or help display
(\pageref{ctrlr}, \pageref{ctrlrr})\\
{\tt CTRL-S}&forward search in \COA\ or help display
(\pageref{ctrls}, \pageref{ctrlss})\\
{\tt CTRL-T}&circulate through task names in \UCA\ (\pageref{ctrlt})\\
{\tt CTRL-U}&clear input line (\pageref{ctrlu})\\
{\tt CTRL-V}&page forward in \COA, help display or keyword display
(\pageref{ctrlv}, \pageref{ctrlvv}, \pageref{ctrlvvv})\\
{\tt CTRL-W}&suspend or resume task (\pageref{ctrlw})\\
{\tt CTRL-X}&initial character for special special combinations\\
{\tt CTRL-Y}&(undefined)\\
{\tt CTRL-Z}&page back in \COA, help display or keyword display
(\pageref{ctrlz}, \pageref{ctrlzz}, \pageref{ctrlzzz})\\
&\\
{\tt CTRL-X E}&edit user-specified file (\pageref{ctrlxe})\\
{\tt CTRL-X S}&switch \COA\ scroll mode on or off (\pageref{ctrlxs})\\
{\tt CTRL-X T}&switch Tektronix emulation on or off (\pageref{ctrlxt})\\
{\tt CTRL-X U}&overlay \H\ with shell (\pageref{ctrlxu})\\
{\tt CTRL-X CTRL-C}&quit \tH\ (\pageref{ctrlxctrlc})\\
\end{tabular}

(Numbers in parentheses refer to pages in this document.)
\chapter{Files used by tHermes}
\begin{tabular}{lll}
\hline
{\bf File name}&{\bf Usage}&{\bf Format}\\
\hline
{\tt GIPSY.LOG}&log file&text file\\
{\tt SCREEN.LOG}&terminal screen pages&internal\\
{\tt HERMES.keywd}&user input parameters (``macro'')&text file\\
{\tt HERMES.cmd}&previous commands (for command recall)&text file\\
{\tt *.def}&default file&text file\\
{\tt *.rcl}&recall file&text file\\
{\tt \$TEKNAME}&Tektronix graphics file (env. variable)&
{\tt PGPLOT} (``{\tt THERTEK}'')\\
{\it taskname\/\tt .key}&scratch file&\\
{\tt screen.prt}&scratch file&\\
{\tt keys.tmp}&scratch file&\\
\hline
\end{tabular}

The file {\tt GIPSY.LOG} contains the log of any number of \G\ sessions and
can be edited and printed. If the information contained by it is not necessary
anymore, it can be deleted.

{\tt SCREEN.LOG} contains the screen pages of any number of \G\
sessions. It cannot be edited or printed. Any modification of this file can
cause \tH\ to fail. If the information contained by it is not necessary
anymore, it can be deleted.

{\tt HERMES.keywd} contains all current user input parameters of
previously run tasks. This file can be edited, provided its format is
maintained. It would however be very seldomly necessary to edit this file.

{\tt HERMES.cmd} contains the last 1000 commands which were typed into
the \UCA. It is used for \tH' command recall feature.
This file can be edited, but this would be very seldomly necessary.

The files with the extension {\tt .def} are the user-defined default files.

The files with the extension {\tt .rcl} are the user-defined recall files.

The file to which the environment variable {\tt TEKNAME} points, contains
information to be displayed on a Tektronix~4010 compatible graphics
terminal. It can only be written by the {\tt PGPLOT} driver for the
``{\tt THERTEK}'' device.

The scratch files have normally all been cleaned up when \tH\ exits. If such
a file is present anyway, it can be deleted without any problem.

\end{document}
